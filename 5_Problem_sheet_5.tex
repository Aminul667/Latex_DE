\documentclass[11pt,a4paper]{article}
\usepackage[margin=1in, headheight=14pt]{geometry}
\usepackage{amsfonts,amsmath,amssymb,suetterl}
\usepackage{lmodern}
\usepackage[T1]{fontenc}
\usepackage{fancyhdr}
\usepackage{float}
\usepackage[utf8]{inputenc}
\usepackage{fontawesome}
\usepackage{enumerate}
\usepackage{xcolor}
\usepackage{hyperref}
\usepackage{tikz}
\usepackage{nicefrac}
\usepackage{subcaption}
\usepackage{physics}
\usepackage{mathtools}
\usepackage{adjustbox}

\DeclareUnicodeCharacter{2212}{-}

\usepackage{mathrsfs}
\usepackage[nodisplayskipstretch]{setspace}

\setstretch{1.5}
\renewcommand{\footrulewidth}{0pt}

\pagestyle{fancy}
\fancyhead[R]{Problem Sheet 5}
\fancyhead[L]{MA202: Differential Equations}

\parindent 0ex
\setlength{\parskip}{1em}
\raggedbottom

\begin{document}
	\begin{center}
		\textbf{Problem Sheet 5}
	\end{center}
	%
	\begin{enumerate}
		\item For the systems of ODEs given below, find the equation of the form $H(x, y) = c$ satisfied by the trajectories. Sketch/plot several level curves of the function $H$ and indicate the direction of motion for each trajectory:\\
		\textbf{(A)} $dx/dt = y,\quad dy/dt = 4x$;\\
		\textbf{(B)} $dx/dt = -x + y,\quad dy/dt = -x - y$;\\
		\textbf{(C)} $dx/dt = -x + y + x^2,\quad dy/dt = y - 2xy$.
		\item Prove that for the system
		$$
		\frac{dx}{dt} = F(x,y),\quad \frac{dy}{dt} = G(x,y)
		$$
		there is at most one trajectory passing through a given point $(x_0, y_0)$.\\
		\textbf{Hint:} Let $C_0$ be the trajectory generated by the solution $x = \phi_0(t),\ y = \psi_0(t)$, with $\phi_0(t_0) = x_0,\ \psi_0(t_0) = y_0$, and let $C_1$ be the trajectory generated by the solution $x = \phi_1(t),\ y = \psi_1(t)$, with $\phi_1(t_1) = x_0,\ \psi_1(t_1) = y_0$. Use the fact that the system is autonomous, and also the existence and uniqueness theorem, to show that $C_0$ and $C_1$ are the same.
		\item Prove that if a trajectory starts at a noncritical point of the system
		$$
		\frac{dx}{dt} = F(x,y),\quad \frac{dy}{dt} = G(x,y),
		$$
		then it cannot reach a critical point (x0, y0) in a finite length of time.\\
		\textbf{Hint:} Assume the contrary; that is, assume that the solution $x = \phi(t),\ y = \psi(t)$ satisfies $\phi(a) = x_0,\ \psi(a) = y_0$. Then use the fact that $x = x_0,\ y = y_0$ is a solution of the given system satisfying the initial condition $x = x_0,\ y = y_0$ at $t = a$.
		\item For the systems of ODEs given below, determine all critical points and find the corresponding linear system near each critical point. Then determine the stability properties of the system around each critical point:\\
		\textbf{(A)} $dx/dt = (3 + x)(y - x),\quad dy/dt = (4 - x)(y + x)$;\\
		\textbf{(B)} $dx/dt = -x - x^2 - xy,\quad dy/dt = 3y - xy - 2y^2$;\\
		\textbf{(C)} $dx/dt = 2x + y + xy^3,\quad dy/dt = x - 2y - xy$;\\
		\textbf{(D)} $dx/dt = (2 + x) \sin y,\quad dy/dt = 1 - x - \cos y$.
		\item Consider the system
		$$
		\frac{dx}{dy} = x(a - \sigma x - \alpha y),\quad \frac{dy}{dt} = -y(-c + \gamma x),
		$$
		where $a,\ \sigma,\ \alpha,\ c$, and $\gamma$ are positive constants.\\
		\textbf{(A)} Find all critical points of the given system. How does their location change as $\sigma$ increases from zero? Assume that $a/\sigma > c/\gamma$, that is, $\sigma < a\gamma/c$. Why is this assumption necessary?\\
		\textbf{(B)} Determine the nature and stability characteristics of each critical point.\\
		\textbf{(C)} Show that there is a value of $\sigma$ between zero and $a\gamma /c$ where the critical point in the interior of the first quadrant changes from a spiral point to a node.\\
		\textbf{(D)} Describe the effect on the two populations as $\sigma$ increases from zero to $a\gamma /c$.
		\item For the systems of ODEs given below, construct a suitable Liapunov function of the form $ax^2 + cy^2$, where $a$ and $c$ are to be determined. Then show that the critical point at the origin is of the indicated type:\\
		\textbf{(A)} $dx/dt = -x^3_2xy^2,\ dy/dt = −2x^2y - y^3$ (asymptotically stable);\\
		\textbf{(B)} $dx/dt = -2x^3 + 2y^3,\ dy/dt = -2xy^2$ (stable);\\
		\textbf{(C)} $dx/dt = 2x^3 - y^3,\ dy/dt = 2xy^2 + 4x^2 2y + 2y^3$ (unstable).
		\item Consider the system of equations
		$$
		\frac{dx}{dt} = y - xf(x,y),\quad \frac{dy}{dx} = -x - yf(x,y),
		$$
		where $f(x, y)$ is continuous and has continuous first partial derivatives. Show that if $f(x, y) > 0$ in some neighborhood of the origin, then the origin is an asymptotically stable critical point, and if $f(x, y) < 0$ in some neighborhood of the origin, then the origin is an unstable critical point.\\
		\textbf{Hint:} Construct a Liapunov function of the form $c(x^2 + y^2)$.
		\item For the autonomous systems expressed in polar coordinates, given below, determine all periodic solutions, all limit cycles, and determine their stability characteristics:\\
		\textbf{(A)} $dr/dt = r^2(4-r^2),\quad d\theta/dt = 1$;\\
		\textbf{(B)} $dr/dt = r(r - 2)(r - 3),\quad d\theta/dt = 1$;\\
		\textbf{(C)} $dr/dt = \sin \pi r,\quad d\theta/dt = 1$.
		\item \textbf{(A)} Show that the system
		$$
		\frac{dx}{dt} = -y + x\frac{f(r)}{r},\quad \frac{dy}{dt} = x + y\frac{f(r)}{r},
		$$
		has periodic solutions corresponding to the zeros of $f(r)$ (Here $r = \sqrt{x^2+y^2}$). What is the direction of motion on the closed trajectories in the phase plane?\\
		\textbf{(B)}  Let $f(r) = r(r - 2)^2(r^2 - 4r + 3)$. Determine all periodic solutions and determine their stability characteristics.
		\item \textbf{(A)} Determine the periodic solutions, if any, of the system
		$$
		\frac{dx}{dt} = y + \frac{x}{\sqrt{x^2 + y^2}}(x^2 + y^2 - 3),\quad \frac{dy}{dt} = -x + \frac{y}{\sqrt{x^2 + y^2}}(x^2 + y^2 - 3).
		$$
		\textbf{(B)} Show that the linear autonomous system
		$$
		\frac{dx}{dt} = a_{11}x + a_{12}y,\quad \frac{dy}{dt} = a_{21}x + a_{22}y
		$$
		does not have a periodic solution (other than $x = 0,\ y = 0$) if $a_{11} + a_{22} \neq 0$.\\
		\textbf{(C)} Show that the given system
		$$
		\frac{dx}{dt} = -3x -3y - xy^2,\quad \frac{dy}{dx} = 2y + x^3 - x^2y
		$$
		has no periodic solutions other than constant solutions.
		\item Consider the system of equations
		\begin{equation}
			\frac{dx}{dt} = \mu x + y - x(x^2 + y^2),\quad \frac{dy}{dt} = -x + \mu y -y(x^2 + y^2),
		\end{equation}
		where $\mu$ is a parameter.\\
		\textbf{(A)} Show that the origin is the only critical point.\\
		\textbf{(B)} Find the linear system that approximates Eqs. (1) near the origin and find its eigenvalues. Determine the type and stability of the critical point at the origin. How does this classification depend on $\mu$?\\
		\textbf{(C)} Rewrite Eqs. (1) in polar coordinates. Then show that when $\mu > 0$, there is a periodic solution $r = \sqrt{\mu}$ and that this periodic solution attracts all other nonzero solutions.\\
		\textbf{Note:} As the parameter µ increases through the value zero, the previously asymptotically stable critical point at the origin loses its stability, and simultaneously a new asymptotically stable solution (the limit cycle) emerges. Thus the point $\mu = 0$ is a bifurcation point; this type of bifurcation is called a Hopf\footnote{Eberhard Hopf (1902-1983) was born in Austria and educated at the University of Berlin but spent much of his life in the United States, mainly at Indiana University. Hopf bifurcations are named for him in honor of his rigorous treatment of them in a 1942 paper.} bifurcation.
		\item Consider the ellipsoid
		$$
		V(x,y,z) = rx^2 + \sigma y^2 + \sigma(z-2r)^2 = c > 0.
		$$
		\textbf{(A)} Calculate dV/dt along trajectories of the Lorenz equations:
		$$
		\frac{dx}{dt} = \sigma(-x + y),\quad \frac{dy}{dt} = rx-y-xz,\quad \frac{dz}{dt} = -bz + xy.
		$$
		\textbf{(B)} Determine a sufficient condition on c so that every trajectory crossing $V(x, y, z) = c$ is directed inward.\\
		\textbf{(C)} Evaluate the condition found in part (B) for the case $\sigma = 10,\ b = 8/3,\ r = 28$.
		%
		\begin{table}[H]
			\centering
			\begin{tabular}{ |l|l|l| } 
			 \hline
			 \textbf{Eigenvalues} & \textbf{Type of Critical Point} & \textbf{Stability}\\
			 \hline
			 \hline
			 $r_1>r_2>0$ & Node & Unstable \\
			 \hline
			 $r_1 < r_2 < 0$ & Node & Asymptotically Stable \\
			 \hline
			 $r_2 < 0 < r_1$ & Saddle Point & Unstable \\ 
			 \hline
			 $r_1 = r_2 > 0$ & Node or spiral point & Unstable \\
			 \hline
			 $r_1 = r_2 < 0$ & Node or spiral point & Asymptotically Stable \\
			 \hline
			 $r_1,r_2 = \lambda \pm i\mu$ & Spiral Point &  \\
			 
			 $\lambda > 0$ &  & Unstable \\
			 
			 $\lambda < 0$ &  & Asymptotically Stable \\
			 \hline
			 $r_1 = i\mu, r_2 = -i\mu$ & Center or spiral point & Indeterminate \\
			 \hline
			\end{tabular}
			\caption{Stability Properties of Almost Linear Systems}
		\end{table}
		%
	\end{enumerate}
\end{document}