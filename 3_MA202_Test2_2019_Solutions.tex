\documentclass[11pt,a4paper]{article}
\usepackage[margin=1in, headheight=14pt]{geometry}
\usepackage{amsfonts,amsmath,amssymb,suetterl}
\usepackage{lmodern}
\usepackage[T1]{fontenc}
\usepackage{fancyhdr}
\usepackage{float}
\usepackage[utf8]{inputenc}
\usepackage{fontawesome}
\usepackage{enumerate}
\usepackage{xcolor}
\usepackage{nicefrac}
\usepackage{physics}
\usepackage{mathtools}
\usepackage{mathrsfs}
\usepackage[nodisplayskipstretch]{setspace}

\DeclareUnicodeCharacter{2212}{-}

\setstretch{1.5}
\renewcommand{\footrulewidth}{0pt}

\pagestyle{fancy}
\fancyhead[R]{Test 2 (Spring 2019) - Solutions}
\fancyhead[L]{MA202: Differential Equations}

\parindent 0ex
\setlength{\parskip}{1em}
\raggedbottom

\newcommand{\mrk}[1]{\hfill\textbf{[#1 marks]}}

\begin{document}
	%
	\begin{center}
		\textbf{\large Test 2 (Spring 2019)}\\[0.15cm]
		\textbf{\large SOLUTIONS}
	\end{center}
	%
	\begin{enumerate}
		\item The characteristic equation is given by
		$$
		\begin{vmatrix}
			1 - r & -i\\
			i & 1 - r
		\end{vmatrix} 
		= (1 - r)^2 + i^2 = r(r - 2) = 0.
		$$
		The equation has roots $r_1 = 0$ and $r_2 = 2$. For $r = 0$, the components of the solution vector must satisfy $\xi_1 − i\xi_22 = 0$. Thus the corresponding eigenvector is $\xi^{(1)} = (1, −i)^T$. Substitution of $r = 2$ results in the single equation $-\xi_1 - i\xi_2 = 0$. A corresponding eigenvector is $\xi^{(2)} = (1, i)^T$. Since the eigenvalues are distinct, the general solution is
		$$
		\vb{z}(t) = c_1
		\begin{pmatrix}
			1\\
			-i
		\end{pmatrix} + c_2
		\begin{pmatrix}
			1\\
			i
		\end{pmatrix}e^{2t}
		$$
		\mrk{10}
		\item The solution to the ODEs is based on the analysis of the algebraic equations
		$$
		\begin{pmatrix}
			2 - r & 1\\
			-5 & -2 - r
		\end{pmatrix}
		\begin{pmatrix}
			\xi_1\\
			\xi_2
		\end{pmatrix} =
		\begin{pmatrix}
			0\\
			0
		\end{pmatrix}.
		$$
		For a nonzero solution, we require that $\det(\vb{A} - r\vb{I}) = r^2 + 1 = 0$. The roots of the characteristic equation are $r_{1,2} = \pm i$. Setting $r = i$, the equations are equivalent to $(2 - i)\xi_1 + \xi_2 = 0$. The eigenvectors are $\xi^{(1)} = (1, -2 + i)^T$ and $\xi^(2) = (1, -2 - i)^T$. Hence one of the complexvalued solutions is given by
		\begin{align*}
			\vb{x}^{(1)}(t)
			&=
			\begin{pmatrix}
				1\\
				-2 + i
			\end{pmatrix} e^{it} =
			\begin{pmatrix}
				1\\
				-2 + i
			\end{pmatrix}(\cos t + i\sin t)\\
			&=
			\begin{pmatrix}
				\cos t\\
				-2\cos t - \sin t
			\end{pmatrix} + i
			\begin{pmatrix}
				\sin t\\
				\cos t - 2\sin t
			\end{pmatrix}.
		\end{align*}
		%
		Therefore the general solution is
		$$
		\vb{x}(t) = c_1
		\begin{pmatrix}
			\cos t\\
			-2\cos t - \sin t
		\end{pmatrix} + c_2
		\begin{pmatrix}
			\sin t\\
			\cos t - 2\sin t
		\end{pmatrix}.
		$$
		The solution may also be written as
		$$
		\vb{x}(t) = c_1
		\begin{pmatrix}
			-2 \cos t + \sin t\\
			5\cos t
		\end{pmatrix} + c_2
		\begin{pmatrix}
			-2\sin t - \cos t\\
			5\sin t
		\end{pmatrix}.
		$$
		\mrk{10}
		%
		\item The characteristic equation reduces to $r^2 + r - 6 = 0$. The roots of the characteristic equation are $r_1 = 2$ and $r_2 = -3$. For $r = 2$, the system of equations reduces to $\xi_1 = 4\xi_2$. The corresponding eigenvector is $\xi^{(1)} = (4, 1)^T$. Substitution of $r = -3$ results in the single equation $\xi_1 + \xi_2 = 0$. A corresponding eigenvector is $\xi^{(2)} = (1, -1)^T$. Hence, we have the two linearly independent solutions
		$$
		\vb{x}^{(1)} =
		\begin{pmatrix}
			4\\
			1
		\end{pmatrix}e^{2t},\quad\text{and} \quad \vb{x}^{(1)} =
		\begin{pmatrix}
			1\\
			-1
		\end{pmatrix}e^{-3t}.
		$$
		Thus
		$$
		\Psi(t) =
		\begin{pmatrix}
			4e^{2t} & -e^{-3t}\\
			e^{2t} & e^{-3t}
		\end{pmatrix}.
		$$
		Imposing the canonical initial condition we get
		$$
		\Psi(0) =
		\begin{pmatrix}
			4 & -1\\
			1 & 1
		\end{pmatrix},\quad \text{and}\quad \Psi^{-1}(0) = \frac{1}{5}
		\begin{pmatrix}
			1 & 1\\
			-1 & 4
		\end{pmatrix},
		$$
		so that
		$$
		\Phi(t) = \Psi(t)\Psi^{-1}(0) = \frac{1}{5}
		\begin{pmatrix}
			e^{-3t} + 4e^{2t} & -4e^{-3t} + 4e^{2t}\\
			-e^{-3t} + e^{2t} & 4e^{-3t} + e^{2t}
		\end{pmatrix}.
		$$
		\mrk{10}
		\item The eigenvalues of
		$$
		\begin{pmatrix}
			2 & 3\\
			-1 & -2
		\end{pmatrix}
		$$
		are given by $r_1 = 1$ and $r_2 = -1$. The corresponding eigenvectors are given by
		$$
		\xi^{(1)} =
		\begin{pmatrix}
			-3\\
			1
		\end{pmatrix},\quad 
		\xi^{(2)} =
		\begin{pmatrix}
			-1\\
			1
		\end{pmatrix}.
		$$
		Therefore, two linearly independent solutions are given by
		$$
		\vb{x}^{(1)} = 
		\begin{pmatrix}
			-3\\
			1
		\end{pmatrix}e^t,\quad
		\vb{x}^{(2)} =
		\begin{pmatrix}
			-1\\
			1
		\end{pmatrix}e^{-t}.
		$$
		and
		$$
		\Psi(t) =
		\begin{pmatrix}
			-3e^t & -e^{-t}\\
			e^t & e^{-t}
		\end{pmatrix}
		$$
		is a fundamental matrix. In order to find the general solution using variation of parameters, we need to calculate $\int^t_{t_1}\Psi^{-1}(s)\vb{g}(s)ds$. We see that
		$$
		\Psi^{-1}(s) = \frac{1}{2}
		\begin{pmatrix}
			-e^{-s} & -e^{-s}\\
			e^s & 3e^s
		\end{pmatrix}
		$$
		Therefore,
		\begin{align*}
			\int_{t_1}^t \Psi^{-1}(s)\vb{g}(s)ds = \frac{1}{2}\int_{t_1}^t
			\begin{pmatrix}
				-e^{-s} & -e^{-s}\\
				e^s & 3e^s
			\end{pmatrix}
			\begin{pmatrix}
				e^s\\
				s
			\end{pmatrix}ds\\
			= \frac{1}{2}\int_{t_1}^t
			\begin{pmatrix}
				-1 -se^{-s}\\
				e^{2s} + 3se^s
			\end{pmatrix}ds = \frac{1}{2}
			\begin{pmatrix}
				-t + te^{-t} + e^{-t}\\
				\frac{1}{2}e^{2t} + 3te^t - 3e^t
			\end{pmatrix} + \vb{c.}
		\end{align*}
		Then the general solution will be given by
		\begin{align*}
			\vb{x}(t)
			&= \Psi(t)\vb{c} + \Psi(t)\int_{t_1}^t \Psi^{-1}(s)\vb{g}(s)ds\\
			&= 
			\begin{pmatrix}
				-3e^t & -e^{-t}\\
				e^t & e^{-t}
			\end{pmatrix}\vb{c}\\
			&+
			\begin{pmatrix}
				-3e^t & -e^{-t}\\
				e^t & e^{-t}
			\end{pmatrix}
			\left[
				\frac{1}{2}
				\begin{pmatrix}
					-t + te^{-t} + e^{-t}\\
					\frac{1}{2}e^{2t} + 3te^t - 3e^t
				\end{pmatrix} + \vb{c}
			\right]\\
			&= c_1e^t
			\begin{pmatrix}
				-3\\
				1
			\end{pmatrix} + c_2e^{-t}
			\begin{pmatrix}
				-1\\
				1
			\end{pmatrix} +
			\begin{pmatrix}
				\left(\frac{3}{2}t - \frac{1}{4}\right)e^t - 3t\\
				\left(-\frac{1}{2}t + \frac{1}{4}\right)e^t + 2t - 1
			\end{pmatrix}.
		\end{align*}
		\mrk{10}
		\item The critical points are given by the solution set of the equations
		\begin{align*}
			(2 + x) \sin y &= 0\\
			1 - x - \cos y = 0.
		\end{align*}
		If $x = -2$, then we must have $\cos y = 3$, which is impossible (for real $y$). Therefore $\sin y = 0$, which implies that $y = n\pi,\ n \in \mathbb{Z}$. Based on the second equation,
		$$
		x = 1 - \cos n\pi.
		$$
		It follows that the critical points are located at $(0, 2k\pi)$ and $(2,(2k +1)\pi)$, where $k \in \mathbb{Z}$.\\
		Given that $F(x, y) = (2 + x) \sin y$ and $G(x, y) = 1 - x - \cos y$, the Jacobian matrix of the vector field is
		$$
		\begin{pmatrix}
			F_x(x,y) & F_y(x,y)\\
			G_x(x,y) & G_y(x,y)
		\end{pmatrix} =
		\begin{pmatrix}
			\sin y & (2 + x)\cos y\\
			-1 & \sin y
		\end{pmatrix}
		$$
		At the critical points $(0, 2k\pi)$, the coefficient matrix of the linearised system is
		$$
		J(0, 2k\pi) =
		\begin{pmatrix}
			0 & 2\\
			-1 & 0
		\end{pmatrix}
		$$
		with purely imaginary eigenvalues $r_{1,2} = \pm \sqrt{2}i$. The critical points of the associated linear systems are centres, which are stable (see Theorem 9.3.2 in the main Textbook). Note that Theorem 9.3.2 does not provide a definite conclusion regarding the relation between the nature of the critical points of the nonlinear systems and their corresponding linearisations.\\
		At the points $(2,(2k + 1)\pi)$, the coefficient matrix of the linearised system is
		$$
		J[2, (2k+1)\pi] =
		\begin{pmatrix}
			0 & -4\\
			-1 & 0
		\end{pmatrix},
		$$
		with eigenvalues $r_1 = 2$ and $r_2 = -2$. The eigenvalues are real, with opposite sign. Hence the critical points of the associated linear systems are saddles, which are unstable.\\
		As asserted in Theorem 9.3.2 in the main Textbook, the trajectories near the critical points $(2,(2k + 1)\pi)$ resemble those near a saddle. Upon closer examination, the critical points $(0, 2k\pi)$ are indeed centres.\par
		\vspace*{0ex}\mrk{10}
	\end{enumerate}
\end{document}