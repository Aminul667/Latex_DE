\documentclass[11pt,a4paper]{article}
\usepackage[margin=1in, headheight=14pt]{geometry}
\usepackage{amsfonts,amsmath,amssymb,suetterl}
\usepackage{lmodern}
\usepackage[T1]{fontenc}
\usepackage{fancyhdr}
\usepackage{float}
\usepackage[utf8]{inputenc}
\usepackage{fontawesome}
\usepackage{enumerate}
\usepackage{xcolor}
\usepackage{nicefrac}
\usepackage{physics}
\usepackage{mathtools}
\usepackage{adjustbox}

\DeclareUnicodeCharacter{2212}{-}

\usepackage{mathrsfs}
\usepackage[nodisplayskipstretch]{setspace}

\setstretch{1.5}
\renewcommand{\footrulewidth}{0pt}

\pagestyle{fancy}
\fancyhead[R]{Sample Test 2}
\fancyhead[L]{MA323: Partial Differential Equations}

\parindent 0ex
\setlength{\parskip}{1em}
\raggedbottom

\newcounter{mk}
\newcommand{\mrk}[1]{\hfill\textbf{[#1 marks]}}
\newcommand{\qtion}{\stepcounter{mk}\textbf{Question \arabic{mk}: }}

\begin{document}
	\begin{enumerate}
		\item Find all separable solutions $u(x, y) = F(x)G(y)$ of the equation
		$$
		xu_y = yu_x
		$$
		\mrk{10}
		\item Consider the equation
		$$
		xu_x + u_y = 1.
		$$
		\begin{enumerate}[(a)]
			\item Find a characteristic curve passing through the point $(1, 1, 1)$.\\
			\vspace*{0ex}\mrk{4}
			\item Find the solution of the equation above $u(x, y)$ satisfying the initial condition
			$$
			u(x, 0) = \sin x.
			$$
			Is the solution defined for all $x$ and $y$?\\
			\vspace*{0ex}\mrk{6}
		\end{enumerate}
		\item Find the region where the equation
		$$
		x^2u_{xx} + 2xyu_{xy} + 2x^2u_{yy} + yu_y = 0.
		$$
		is elliptic, and then reduce it to its canonical form in this region.\\
		\vspace*{0ex}\mrk{10}\\
		\textbf{Hint:} You may find useful the formulae given on the last sheet of the test paper.
		\item Find the Fourier series of the given function
		$$
		f(x) = 
		\begin{cases}
			\cos x, & 0\leq x \leq \pi\\
			0, & \pi \leq x \leq 2\pi.
		\end{cases}
		$$
		\mrk{10}
		\item Solve the boundary value problem for the Laplace equation
		\begin{align*}
			&u_{xx} + u_{yy} = 0, & 0 < x < a, &\ 0 < y < b\\
			&u(x, 0) = \sin(\frac{\pi x}{a}), & 0 < x < a &\\
			&u(x, b) = 0, & 0 < x < a &\\
			&u(0, y) = u(a, y) = 0, & 0 < y < b &
		\end{align*}
		\mrk{10}
	\end{enumerate}
	\pagebreak
	\thispagestyle{empty}
	\rule{\textwidth}{0.4pt}
	\begin{center}
		\textbf{Formulae}
	\end{center}
	Consider the invertible transformation of the independent variables
	$$
	\xi = \xi(x, y),\quad \eta = \eta(x, y),\quad \frac{\partial(\xi, \eta)}{\partial(x,y)}\neq 0.
	$$
	If $\tilde{u}(\xi, \eta)$ is the transformation of $u(x, y)$, i.e. $u(x, y) = \tilde{u}(\xi(x, y),\ \eta(x, y))$, then
	\begin{align*}
		u_x &= \tilde{u}_\xi\xi_x + \tilde{u}_\eta\eta_x,\\
		u_y &= \tilde{u}_\xi\xi_y + \tilde{u}_\eta\eta_y,\\
		u_{xx} & =\tilde{u}_{\xi\xi}\xi^2_x + \tilde{u}_{\eta\eta}\eta_x^2 + 2\tilde{u}_{\xi\eta}\xi_x\eta_x + \tilde{u}_\xi\xi_{xx} + \tilde{u}_\eta\eta_{xx},\\
		u_{xy} &= \tilde{u}_{\xi\xi}\xi_x\xi_y + \tilde{u}_{\eta\eta}\eta_x\eta_y + \tilde{u}_{\xi\eta}(\xi_x\eta_y + \xi_y\eta_x) + \tilde{u}_\xi\xi_{xy} + \tilde{u}_\eta\eta_{x,y},\\
		u_{yy} & =\tilde{u}_{\xi\xi}\xi^2_y + \tilde{u}_{\eta\eta}\eta_y^2 + 2\tilde{u}_{\xi\eta}\xi_y\eta_y + \tilde{u}_\xi\xi_{yy} + \tilde{u}_\eta\eta_{yy}.
	\end{align*}
	\vfill\centering\textbf{END OF PAPER}
\end{document}