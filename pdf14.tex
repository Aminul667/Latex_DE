\documentclass[11pt,a4paper,twoside]{article}
\usepackage[margin=1in, headheight=14pt]{geometry}
\usepackage{amsfonts,amsmath,amssymb,suetterl}
\usepackage{lmodern}
\usepackage[T1]{fontenc}
\usepackage{fancyhdr}
\usepackage{float}
\usepackage[utf8]{inputenc}
\usepackage{fontawesome}
\usepackage{enumerate}
\usepackage{xcolor}
\usepackage{hyperref}
\usepackage{physics}

\DeclareUnicodeCharacter{2212}{-}

\usepackage{mathrsfs}
\usepackage[nodisplayskipstretch]{setspace}

\setstretch{1.5}
\renewcommand{\footrulewidth}{0pt}

\parindent 0ex
\setlength{\parskip}{1em}

\begin{document}
    %
	\begin{singlespace}
		\begin{center}
			\Huge Queen Mary\\
			\LARGE University of London
		\end{center}
		\Large \textbf{MTH5123} \hfill \Large \textbf{Differential Equations,} \hfill \Large \textbf{Autumn 2020}\\
		\large \textbf{Coursework 2 \- Week 4 – Selected Solutions} \hfill \large \textbf{W. Huang}
		\rule{\textwidth}{0.4pt}
	\end{singlespace}
	%
	\textbf{I. Practice Problems}\par
	\textbf{A.} Find the general solutions of the following linear homogeneous differential equations of second order:
	\begin{enumerate}
		\item $y^{\prime\prime}+y^\prime - 12y=0:$ The characteristic equation is $\lambda^2+\lambda-12=0$ which has two real roots $\lambda_1 = 3,\ \lambda_2 = -4$, hence the general solution is $y_h = C_1e^{3x}+C_2e^{-4x}$.
		\item $6y^{\prime\prime} + 5y^\prime -6y = 0$: The characteristic equation is $6\lambda^2+5\lambda-6=0$ which has two real roots $\lambda_1 = 2/3,\ \lambda_2-3/2$, hence the general solution is $y_h = C_1e^{\frac{2}{3}x}+C_2e^{-\frac{3}{2}x}$.
		\item $y^{\prime\prime} + 2y^\prime+17y=0$: The characteristic equation is $\lambda^2+2\lambda+17=0$ which has two complex conjugate roots $\lambda_1 = -1+4i,\ \lambda_2 = -1-4i$,  hence the general solution can be written as $y_h = e^{-x}(C_1e^{4ix}+C_2e^{-4ix})$ where $C_{1,2}$ are two complex constants, or equivalently as $y_h=e^{-x}(A\cos(4x)+B\sin(4x))$ where $A,\ B$ are two real constants.
		\item $y^{\prime\prime} + 2y^\prime + 3y = 0$: The characteristic equation is $\lambda^2 + 2\lambda + 3 = 0$ which has two complex conjugate roots $\lambda_1 = -1+i\sqrt{2},\ \lambda_2 = -1-i\sqrt{2}$,  hence the general solution can be written as $y_h = e^{-x}\left(C_1e^{i\sqrt{2}x} + C_2e^{-i\sqrt{2}x}\right)$ where $C{1,2}$ are two complex constants or equivalently as $y_h = e^{-x}(A\cos(\sqrt{x}) + B\sin(\sqrt{2}x))$ where $A,\ B$ are two real constants.
		\item $16y^{\prime\prime} + 8y^\prime + y = 0$: The characteristic equation is $16\lambda^2+8\lambda+1=0$ which has a real root $\lambda = −1/4$ of multiplicity two, hence the general solution can be written as $y_h = e^{-x/4}(C_1+C_2x)$.
	\end{enumerate}
	%
	\textbf{B.} Solve the following initial value problems:
	\begin{enumerate}
		\item $10y^{\prime\prime} - y^\prime - 3y = 0,\ y(0) = 1,\ y^\prime(0) = 0$: The characteristic equation is $10\lambda^2-\lambda-3=0$ which has two real roots $\lambda_1 = 3/5,\ \lambda_2 = -1/2$, hence the general solution is $y_h = C_1e^{\frac{3}{5}x}+C_2e^{-\frac{1}{2}x}$.\\
		Taking the derivative: $y^\prime_h = \frac{3}{5}C_1e^{\frac{3}{5}x} - \frac{1}{2}C_2e^{-\frac{1}{2}x}$. We then have $y(0) = C_1 + C_2 = 1,\ y^\prime(0) = \frac{3}{5}C_1-\frac{1}{2}C_2 = 0$. One can solve it by various methods. The most general method  (although perhaps not the easiest) is to rewrite the system of linear equations in the matrix form as $A\vb{c} = \vb{b}$ where
		$$
		A
		=
		\begin{pmatrix}
			1 & 1\\
			\frac{3}{5} & -\frac{1}{2}
		\end{pmatrix},\ 
		\vb{c}
		=
		\begin{pmatrix}
			C_1\\
			C_2
		\end{pmatrix},\ 
		\vb{b}
		=
		\begin{pmatrix}
			1\\
			0
		\end{pmatrix}
		$$
		and solve it as
		$$
		\vb{c} = A^{-1}\vb{b} = \frac{1}{-\frac{1}{2}-\frac{3}{5}}
		\begin{pmatrix}
			-\frac{1}{2} & -1\\
			-\frac{3}{5} & 1
		\end{pmatrix}
		\begin{pmatrix}
			1\\
			0
		\end{pmatrix}
		=
		\begin{pmatrix}
			\frac{5}{11}\\
			\frac{6}{11}
		\end{pmatrix}
		$$
		so that
		$C_1 = \frac{5}{11},\ C_2 = \frac{6}{11}$. Finally, the solution to the initial value problem is given by $y = \frac{5}{11}e^{\frac{3}{5}x}+\frac{6}{11}e^{-\frac{1}{2}x}$.
		\item $y^{\prime\prime} - 2y^\prime - 3y = 0,\ y(0)=2,\ y^\prime(0) = -3$: The characteristic equation is $\lambda^2 - 2\lambda -3 = 0$ which has two real roots $\lambda_1 = 3,\ \lambda_2 = −1$, hence the general solution is $y_h = C_1e^{3x} + C_2e^{-x}$. Taking the derivative: $y^\prime_h = 3C_1e^{3x} - C_2e^{-x}$. We then have $y(0) = C_1 + C_2 = 2,\ y^\prime(0) = 3C_1 - C_2 = -3$.  One can solve these equations by various methods to find $C_1 = -\frac{1}{4},\ C_2 = \frac{9}{4}$.  Finally, the solution to the initial value problem is given by $y = -\frac{1}{4}e^{3x}+\frac{9}{4}e^{-x}$.
		\item $y^{\prime\prime} -4y^\prime -5y = 0,\ y(0) = -1,\ y^\prime(0) = -1$: The characteristic equation is $\lambda^2-4\lambda - 5 = 0$ which has two real roots $\lambda_1 = 5,\ \lambda_2 = -1$, hence the general solution is $y_h = C_1e^{5x}+C_2e^{-x}$. Taking the derivative: $y_h^\prime = 5C_1e^{5x} - C_2e^{-x}$. We then have $y(0) = C_1+C_2 = -1,\ y^\prime(0) = 5C_1-C_2=-1$. One can solve these equations by various methods to find $C_1 = -\frac{1}{3},\ C_2 = -\frac{2}{3}$. Finally, the solution to the initial value problem is given by $y = -\frac{1}{3}e^{5x}-\frac{2}{3}e^{-x}$.
		\item $y^{\prime\prime} - 4y^\prime + 13y = 0,\ y(0) = 4,\ y^\prime(0) = 0$. The characteristic equation is $\lambda^2-4\lambda+13=0$ which has two complex conjugate roots $\lambda_1 = 2+3i,\ \lambda_2 = 2-3i$, hence the general solution is $y_h = C_1e^{(2+3i)x} + C_2e^{(2-3i)x}$ or equivalently $y_h = e^{2x}(A\cos(3x) + B\sin(3x))$ where $A,\ B$ are two real constants. Taking the derivative:
		\begin{equation*}
			\begin{aligned}
				y^\prime_h
				&= 2e^{2x}(A\cos(3x)+B\sin(3x)) + e^{2x}(-3A\sin(3x) + 3B\cos(3x))\\
				&= e^{2x}((2A+3B)\cos(3x) + (2B-3A)\sin(3x))
			\end{aligned}
		\end{equation*}
		Hence $y(0) = A = 4$ and $y^\prime(0) = 2A+3B = 0$, so that $A = 4,\ B = −8/3$ and $y=e^{2x}(4\cos(3x) - \frac{8}{3}\sin(3x))$.
	\end{enumerate}
	%
	\textbf{C.} Assign to each of the following linear homogeneous differential equations
	\begin{enumerate}[\bfseries 1)]
		\item $2y^{\prime\prime} - 8y^\prime + 8y = 0$
		\item $y^{\prime\prime} + y^\prime - 2y = 0$
		\item $y^{\prime\prime} + 2y^\prime + 2y = 0$
	\end{enumerate}
	a correct solution from the list:
	\begin{enumerate}[\bfseries i)]
		\item $y = e^{-x}(2\cos x - \sqrt{2}\sin x)$
		\item $y = e^x + \frac{1}{7}e^{-2x}$
		\item $y = e^{2x}(x+1)$.
	\end{enumerate}
	\textbf{Solution:}
	%
	\begin{enumerate}[\bfseries 1)]
		\item $2y^{\prime\prime} - 8y^\prime + 8y = 0$ corresponds to $y = e^{2x}(x+1)$ (characteristic equation is $2\lambda^2 - 8\lambda + 8 = 2(\lambda - 2)^2 = 0$ with a real root of multiplicity two: $\lambda_1 = \lambda_2 = 2$).
		\item $y^{\prime\prime} + y^\prime - 2y = 0$ corresponds to $y = e^x + \frac{1}{7}e^{-2x}$ (characteristic equation is $\lambda^2 + \lambda -2 = 0$ with two real roots $\lambda_1 = 1,\ \lambda_2 = -2$).
		\item $y^{\prime\prime} + 2y^\prime + 2y = 0$ corresponds to $y = e^{-x}(2\cos x - \sqrt{2}\sin x)$ (characteristic equation is $\lambda^2 + 2\lambda + 2 = 0$ with a pair of complex conjugate roots $\lambda_1 = -1+i,\ \lambda_2 = -1-i$).
	\end{enumerate}
	%
	\textbf{D.} Determine the general solution for the homogeneous linear differential equation
	$$
	y^{\prime\prime} - 2y + y = 0.
	$$
	\textbf{Solution:} The characteristic equation is $\lambda^2 - 2\lambda + 1 = 0$. It has a single root $\lambda = 1$ multiplicity two. Hence the general solution is given by $y_h = (C_1x + C_2)e^x$, which gives after differentiation $y^\prime_h = (C_1x + C_1 + C_2)e^x$. The initial conditions give $y(2) = (2C_1 + C_2)e^2 = 1,\ y^\prime(2) = (3C_1 + C_2)e^2 = -2$. The easiest way to solve this system is to subtract the first equation from the second one, which gives $C_1e^2 = -3$ so that $C_1 = -3/e^2$ and from the first equation $C_2e^2 = 1-2C_1e^2 = 1 + 6 = 7$, hence $C_2 = 7/e^2$. The explicit solution to the initial value problem is $y = (-3x+7)e^{x-2}$.\par
	\textbf{II. Homework Problems}\par
	Solutions will be discussed in Week 8, session 4 (see module session schedule in Qmplus as well).\par
	\textbf{III. More Practice with 2nd Order Linear ODEs}\par
	\textbf{A.} In each exercise below, solve the initial value problem and determine the value of $\alpha$ (if any) so that the solution approaches zero as $t \to \infty$. Sketch/Graph the solution curve.
	%
	\begin{enumerate}
		\item $\ddot{y} + 5\dot{y} + 6y = 0,\ y(0) = \alpha,\ \dot{y}(0) = 3$: Any $\alpha$ will work, since $\lambda = -2,\ -3$. One possible choice for $\alpha$ is $\alpha = 2$, which gives $y(t) = 9e^{-2t} - 7e^{-3t}$.
		\item $\ddot{y} + (2\alpha-1)\dot{y} + \alpha(\alpha-1)y = 0:\ y(t)\to 0$ as long as $\alpha > 1$.
	\end{enumerate}
	%
	\textbf{B.} Study the equation $a\ddot{y} + b\dot{y} + cy = f$, where $a,\ b, c$ and $f$ are all constants. The equilibrium solution is $y_{eq} = f/c$ and the differential equation solved by $Y = y - y_{eq}$ (the deviation from equilibrium) is given by $a\ddot{Y} + b\dot{Y} + cY = 0$.
\end{document}