\documentclass[11pt,a4paper]{article}
\usepackage[margin=1in, headheight=14pt]{geometry}
\usepackage{amsfonts,amsmath,amssymb,suetterl}
\usepackage{lmodern}
\usepackage[T1]{fontenc}
\usepackage{fancyhdr}
\usepackage{float}
\usepackage[utf8]{inputenc}
\usepackage{fontawesome}
\usepackage{enumerate}
\usepackage{xcolor}
\usepackage{nicefrac}
\usepackage{physics}
\usepackage{mathtools}
\usepackage{adjustbox}

\DeclareUnicodeCharacter{2212}{-}

\usepackage{mathrsfs}
\usepackage[nodisplayskipstretch]{setspace}

\setstretch{1.5}
\renewcommand{\footrulewidth}{0pt}

\pagestyle{fancy}
\fancyhead[R]{Test 2 (Spring 2018)}
\fancyhead[L]{MA202: Differential Equations}

\parindent 0ex
\setlength{\parskip}{1em}
\raggedbottom

\newcounter{mk}
\newcommand{\mrk}[1]{\hfill\textbf{[#1 marks]}}
\newcommand{\qtion}{\stepcounter{mk}\textbf{Question \arabic{mk}: }}

\begin{document}
	%
	\begin{center}
		\textbf{\large Test 2 (Spring 2018)}
	\end{center}
	%	
	\fbox{\parbox{15.65cm}{
		\begin{itemize}
			\item Attempt all five questions!
			\item Time allowed: 1 hour.
			\item All questions carry equal marks.
		\end{itemize}
	}}\\[1cm]
	%
	\qtion (a) Given the power-series $y(x) = \sum_{n = 0}^\infty a_nx^n$ of $x$ about $x = 0$, determine $y^\prime(x)$.\\
	\vspace*{0ex}\mrk{3}\\
	(b) Consider the following first-order differential equation and a power-series of $x$ about $x = 0$ as in (a):
	$$
	(1 + x)y^\prime = -y
	$$
	By substituting the power-series and its derivative in (a), determine the coefficients an such that $y(x)$ is a power-series solution to the differential equation.\mrk{7}\\
	\qtion Find the power-series solution to the following differential equation about $x = 0$:
	$$
	(1 - x)y^{\prime\prime} + y = 0.
	$$
	\textbf{Hint:} Follow a similar approach as in Q1 above.\mrk{10}\\
	\qtion Transform the following non-homogeneous system of equations into a single non-homogeneous differential equation of second order:
	$$
	\vb{x}^\prime =
	\begin{pmatrix}
		1 & 1\\
		4 & -2
	\end{pmatrix}\vb{x} +
	\begin{pmatrix}
		2\\
		-1
	\end{pmatrix}e^t.
	$$
	\textbf{Hint:} Write the above system in the form of 2 differential equations for $\dot{x}_1$ and $\dot{x}_2$, and eliminate $x_2$.\mrk{10}\\
	\qtion Consider the system of linear differential equations $x^\prime = \vb{Ax}$, with:
	$$
	\vb{A} =
	\begin{pmatrix}
		-1/2 & 1\\
		-1 & -1/2
	\end{pmatrix}.
	$$
	\begin{enumerate}[(a)]
		\item Calculate the eigenvalues and corresponding eigenvectors of $\vb{A}$.\mrk{2}
		\item Determine the stability of the origin.\mrk{2}
		\item  Write the general solutions to the system of equations.\mrk{4}
		\item  Sketch the phase-portrait of the system.\mrk{2}
	\end{enumerate}
	\qtion Consider the system of linear differential equations $x^\prime = \vb{Ax}$, with
	$$
	\vb{A} =
	\begin{pmatrix}
		2 & -1\\
		3 & -2
	\end{pmatrix}.
	$$
	\begin{enumerate}[(a)]
		\item Calculate the eigenvalues and corresponding eigenvectors of $\vb{A}$.\mrk{3}
		\item Determine the fundamental solutions to the system.\mrk{3}
		\item Construct the fundamental matrix $\Psi(t)$ using the fundamental solutions in (b).\mrk{2}
		\item Find the fundamental matrix $\Phi(t)$ that satisfies $\Phi(0) = I$ for this system, using $\Psi(t)$ from (c).\mrk{2}
	\end{enumerate}
	\vfill\centering\textbf{END}
\end{document}