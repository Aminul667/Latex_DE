\documentclass[11pt,a4paper,twoside]{article}
\usepackage[margin=1in, headheight=14pt]{geometry}
\usepackage{amsfonts,amsmath,amssymb,suetterl}
\usepackage{lmodern}
\usepackage[T1]{fontenc}
\usepackage{fancyhdr}
\usepackage{float}
\usepackage[utf8]{inputenc}
\usepackage{fontawesome}
\usepackage{enumerate}

\usepackage{tikz}
\usetikzlibrary{patterns, arrows.meta}

\DeclareUnicodeCharacter{2212}{-}

\usepackage{mathrsfs}
\usepackage[nodisplayskipstretch]{setspace}

\setstretch{1.5}
\renewcommand{\footrulewidth}{0pt}

\parindent 0ex
\setlength{\parskip}{1em}

\fancyfoot{}
\raggedbottom

% title page
\long\def\mytitle{
	\begin{titlepage}
		\begin{center}
			\Huge Queen Mary\\
			\LARGE University of London
		\end{center}

		\vspace*{\stretch{1}}

		\begin{singlespace}
			{\centering
				{\huge\bfseries MTH5123 Differential Equations\\}
				\vspace{0.5cm}

				{\Large Lecture Notes\\}
				\vspace{0.5cm}

				{\Large Week 2}

				\vfill
				\LARGE Weini Huang
				\vspace{0.5cm}
				
				\LARGE School of Mathematical Sciences\\
				\LARGE Queen Mary University of London\\

				\vspace{0.5cm}
				\LARGE Autumn 2020\\
				}
		\end{singlespace}
	\end{titlepage}
}

\begin{document}
	\pagestyle{empty}
	\mytitle
	%
	Consider equations of the type
	\begin{equation}\label{eq:1}
		y^\prime = \left(\frac{y}{x}\right)
	\end{equation}
	Such ODEs do not change if we rescale $x \to kx$ and $y \to ky$ for any real constant factor $k \ne 0$, hence they are known under the name \textbf{scale-invariant} first order ODEs. To reduce them to separable equations one introduces a new function $z(x) = y(x)/x$ which implies $y(x) = xz(x)$. Differentiating this equation gives $y^\prime = z(x) + xz^\prime(x),$ and \ref{eq:1} can be rewritten in the form $z + xz^\prime = F(z)$ or equivalently
	\begin{equation}\label{eq:2}
		z^\prime = \frac{1}{x}[F(z) - z]
	\end{equation}
	which is indeed separable.\par
	\textbf{Example:}\\
	Solve the equation
	$$
	xy^\prime = y - xe^{y/x}.
	$$
	\textbf{Solution:}\\
	After dividing both sides by $x$ we see that the equation is of the form \ref{eq:1} with the right-hand side $F(z) = z − e^z$. Therefore it is equivalent to the separable equation
	$$
	z^\prime = - \frac{1}{x}e^z.
	$$
	Solving it by standard means leads to $e^{-z} = \ln|x| + C$ or
	$$
	z = -\ln(\ln|x|+C).
	$$
	Finally the general solution to the original ODE is
	$$
	y(x) = -x\ln(\ln|x| + C).
	$$
	%
	%command for section, subsection and page on header
	\numberwithin{equation}{section}
	\pagestyle{fancy}
	\fancyhead[LE,RO]{\thepage}
	\fancyhead[RE]{\nouppercase \leftmark}
	\fancyhead[LO]{\nouppercase \rightmark}

	\section{First order linear ODEs}
	%
	This class of equations is given by
	%
	\begin{equation}\label{eq:1.10}
		y^\prime = A(x) y + B(x),
	\end{equation}
	%
	where the two functions $A(x) \ne 0$ and $B(x)$ are known. These equations are called \textbf{linear} (in $y$), because $y$ and its derivative $y^\prime$ occur only to the first power, they are not multiplied together, nor do they appear as the argument of a function (such as $\sin y,\ \exp(y)$, etc.). If $B(x) = 0$, the equation is called \textbf{homogeneous}, if $B(x) \ne 0$ it is called \textbf{inhomogeneous}.\par
	\textbf{Example:}\\
	$y^\prime \sin(x)y \quad \text{homogeneous}$\\
	$y^\prime e^xy + x \quad \text{inhomogeneous}$\\
	$y^\prime 1 - y^2 + x \quad \text{nonlinear}$\par
	The method of solution of such equations proceeds in two steps:\par
	\textbf{Step 1:} Solve the \textit{homogeneous} equation $y^\prime = A(x)y$, which is separable. The general solution is found to be
	%
	\begin{equation}\label{eq:1.11}
		\int \frac{dy}{y} = \int A(x)dx + C \Rightarrow \ln|y| = \int A(x)dx + C
	\end{equation}
	%
	and finally
	%
	\begin{equation}\label{eq:1.12}
		y = De^{\int A(x)dx},
	\end{equation}
	%
	where $D \in \mathbb{R}$ is an arbitrary real constant (also called a free parameter).\par
	\textbf{Step 2} is known as the variation of parameter method. It amounts to looking for the solution of (\ref{eq:1.10}) in the form
	%
	\begin{equation}\label{eq:1.13}
		y = D(x)e^{\int A(x)dx},
	\end{equation}
	%
	where $D(x)$ is now an unknown function to be determined by substituting (\ref{eq:1.13}) to (\ref{eq:1.10}). This gives
	$$
	y^\prime
	= D^\prime(x)e^{\int A(x)dx} + A(x)D(x)e^{\int A(x)dx}
	= A(x)D(x)e^{\int A(x)dx} + B(x),
	$$
	which after cancelling equal terms on both sides is equivalent to
	%
	\begin{equation}\label{eq:1.14}
		D^\prime e^{\int A(x)dx} = B(x).
	\end{equation}
	%
	This allows us to write $D^\prime (x) = e^{-\int A(x)dx}B(x)$ and to recover $D(x)$ by simple integration
	%
	\begin{equation}\label{eq:1.15}
		D(x) = \int e^{-\int A(x)dx}B(x)dx + C
	\end{equation}
	%
	finally yielding the general solution of (\ref{eq:1.10}) in the form
	%
	\begin{equation}
		y(x) = e^{\int A(x)dx}\left(\int e^{-A(x)dxB(x)dx} + C\right)\quad \forall C \in \mathbb{R}
	\end{equation}
	%
	\textbf{Note:} An alternative method to derive the same result is the \textit{integrating factor method}, as you have seen in Calculus 2.\par
	\textbf{Example:}\\
	Solve the equation
	$$
	y^\prime + 2xy = x
	$$
	\textbf{Solution:}\\
	First we solve $y^\prime + 2xy = 0$ by separation of variables obtaining $y = De^{-x^2}$, where $D$ is an arbitrary constant. Now we assume $D = D(x)$ and substitute $y = D(x)e^{-x^2}$ to the full non-homogeneous equation:
	$$
	y^\prime = D^\prime (x)e^{-x^2} + D(x)(-2x)e^{-x^2}.
	$$
	Thus, we have
	$$
	D^\prime(x)e^{-x^2} + D(x)(-2x)e^{-x^2} + 2xD(x)e^{-x^2} = x
	$$
	which implies $D^\prime(x) = xe^{x^2}$, hence $D(x) = \int xe^{x^2}dx = \frac{1}{2}e^{x^2}+C$. Finally, the general solution to the original ODE is given by
	$$
	y(x)
	= \left(\frac{1}{2}e^{x^2}+C\right)e^{-x^2}
	= \frac{1}{2} + Ce^{-x^2}.
	$$
	
	\section{Exact first order ODEs.}
	%
	\textbf{Exact ODEs} are of the form
	%
	\begin{equation}\label{1.17}
		P(x, y) + Q(x, y)\frac{dy}{dx} = 0.
	\end{equation}
	%
	We would like to find solutions of this class of ODEs in \textit{implicit form} $F(x, y) = C,\ y = y(x)$, for a constant $C$. Using the chain rule we observe that
	%
	\begin{equation}\label{1.18}
		\frac{dF(x, y(x))}{dx}
		= \frac{\partial F}{\partial x} + \frac{\partial F}{\partial y}\frac{dy}{dx}
		= 0,
	\end{equation}
	%
	which coincides with (\ref{1.17}) if we define
	%
	\begin{equation}\label{1.19}
		P(x,y) = \frac{\partial F}{\partial x}, \quad
		Q(x,y) = \frac{\partial F}{\partial y}.
	\end{equation}
	%
	Using these definitions we have
	%
	\begin{equation}\label{1.20}
		\frac{\partial}{\partial y}P(x,y) = \frac{\partial^2F}{\partial y\partial x}, \quad
		\frac{\partial}{\partial x}Q(x,y) = \frac{\partial^2F}{\partial x\partial y}.
	\end{equation}
	%
	If F is twice differentiable in both $x$ and $y$ with continuous second order partial derivatives, we have (according to the \textit{mixed derivatives theorem} in Calculus 2)
	%
	$$
		\frac{\partial^2 F}{\partial y \partial x} = \frac{\partial^2 F}{\partial x \partial y},
	$$
	%
	and we conclude that the equation
	%
	\begin{equation}\label{1.21}
		\frac{\partial }{\partial y}P(x,y) = \frac{\partial}{\partial x}Q(x,y)
	\end{equation}
	%
	must hold. Equation (\ref{1.21}) is the crucial condition for (\ref{1.17}) to be exact. For any exact ODE the general solution can always be written in the implicit form $F(x, y) = C$.\\
	To determine the form of the function $F(x, y)$, one may start with the first equation in (\ref{1.19}) by integrating it over the variable $x$ to
	%
	\begin{equation}\label{1.22}
		Q(x,y)
		= \frac{\partial F}{\partial y}\quad \Rightarrow
		F(x,y)
		= \int P(x,y)dx + g(y), 
	\end{equation}
	%
	where the function $g(y)$ is an arbitrary function of the variable $y$, yet to be determined. To find $g(y)$ we use the second equation in (\ref{1.19})
	%
	\begin{equation}\label{1.23}
		Q(x,y)
		= \frac{\partial F}{\partial y}
		= \frac{\partial}{\partial y}\int P(x,y)dx + g^\prime(y),
	\end{equation}
	%
	which gives
	%
	\begin{equation}\label{1.24}
		g^\prime(y)
		= Q(x,y) - \frac{\partial}{\partial y}\int P(x,y)dx
	\end{equation}
	%
	The missing function $g(y)$ can then be found by straightforward integration of this equation.\par
	\textbf{Example:}\\
	Show that the equation
	$$
	3x^2 + y - (3y^2-x)\frac{dy}{dx} = 0
	$$
	is exact and find its general solution in implicit form.\par
	\textbf{Solution:}\\
	We identify $P(x, y) = 3x^2 + y$, hence $\frac{\partial P}{\partial y} = 1$. Similarly, $Q(x,y) = -(3y^2-x)$, hence $\frac{\partial Q}{\partial x} = 1$. Since $\frac{\partial P}{\partial y} = \frac{\partial Q}{\partial x}$ the equation is exact.\\
	We find its implicit solution in the form $F(x, y) = C$ by
	$$
	F(x,y)
	= \int P(x,y)dx + g(y)
	= \int (3x^2 + y)dx + g(y)
	= x^3 + xy + g(y),
	$$
	where $g(y)$ is yet undetermined. We further have
	$$
	\frac{\partial F}{\partial y}
	= x + g^\prime(y)
	= Q(x,y)
	= -(2y^2 - x), \Rightarrow g^\prime(y) = -3y^2.
	$$
	This allows us to find
	$$
	g(y) = \int (-3y^2)dy = -y^3 + C_1,
	$$
	where $C_1$ is an arbitrary constant. There is no need to keep $C_1$, as it can always be absorbed into the constant $C$. The general solution of the original equation in implicit form is obtained as
	$$
	F(x,y) = x^3+xy-y^3 = C.
	$$
	\textbf{Note:}\\
	The same ODE can be presented in a different form, for example:
	$$
	\frac{dy}{dx} = \frac{3x^2+y}{3y^2-x}
	$$
	One needs to recognize the equivalence of this equation to the form of an exact ODE by then applying the same procedure for a solution.
\end{document}