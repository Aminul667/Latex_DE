\documentclass[11pt,a4paper]{article}
\usepackage[margin=1in, headheight=14pt]{geometry}
\usepackage{amsfonts,amsmath,amssymb,suetterl}
\usepackage{lmodern}
\usepackage[T1]{fontenc}
\usepackage{fancyhdr}
\usepackage{float}
\usepackage[utf8]{inputenc}
\usepackage{fontawesome}
\usepackage{enumerate}
\usepackage{xcolor}
\usepackage{nicefrac}
\usepackage{subcaption}
\usepackage{physics}
\usepackage{mathtools}
\usepackage{adjustbox}

\DeclareUnicodeCharacter{2212}{-}

\usepackage{mathrsfs}
\usepackage[nodisplayskipstretch]{setspace}

\setstretch{1.5}
\renewcommand{\footrulewidth}{0pt}

\pagestyle{fancy}
\fancyhead[R]{Problem Sheet 3 (Solutions)}
\fancyhead[L]{MA323: Partial Differential Equations}

\parindent 0ex
\setlength{\parskip}{1em}
\raggedbottom

\begin{document}
	\begin{center}
		\textbf{Problem Sheet 3 - (Solutions)}
	\end{center}
	%
	\begin{enumerate}
		\item For $m = 0$, we have
		$$
		A_0 = \frac{1}{\pi}\int_{-\pi}^\pi |\sin x|dx = \frac{4}{\pi}.
		$$
		For $m > 0$,
		$$
		A_m = \frac{1}{\pi}\int_{-\pi}^\pi |\sin x|\cos mx dx = \frac{2}{\pi}\int_0^\pi\sin x\cos mx dx.
		$$
		Now
		$$
		\sin a \cos b = \frac{1}{2}[\sin(b+a) - \sin(b - a)],
		$$
		so the integral becomes
		\begin{align*}
			\frac{1}{\pi}\int_0^\pi[\sin(b + a) - \sin(b - a)]dx
			&= \frac{1}{\pi}\left[-\frac{1}{1 + m}\cos (m + 1)x + \frac{1}{1 - m}\cos(m - 1)x\right]_0^\pi\\
			&=
			\begin{cases}
				0 & m - \text{odd},\\
				\frac{-4}{\pi(m^2 - 1)} & m - \text{even}.
			\end{cases}
		\end{align*}
		Thus
		$$
		|\sin x| = \frac{2}{\pi} + \sum_{m-\text{even}} \frac{-4}{\pi(m^2 - 1)}\cos mx.
		$$
		Evaluating this at $x = 0$ and letting $m = 2n$ gives
		$$
		\sum_{n = 1}^\infty\frac{1}{4n^2 - 1} = \frac{1}{2}.
		$$
		Evaluating at $x = \pi/2$ gives
		$$
		1 = \frac{2}{\pi} + \sum_{n = 1}^\infty \frac{-4(-1)^n}{\pi(4n^2 - 1)},
		$$
		so
		$$
		\sum_{n=1}^\infty \frac{-4(-1)^n}{\pi(4n^2 - 1)} = \frac{2 - \pi}{4}.
		$$
		\item The solution of $X^\prime = \lambda X$ is $X = Ce^{\lambda x}$. The boundary condition $X(0) = X(1)$ therefore implies $e^\lambda = 1$, so $\lambda = 2n\pi i$. The eigenfunctions are $X_n = e^{2n\pi ix}$, and since
		$$
		(X_m, X_n) = \int_0^1 X_m(x)\overline{X_n(x)}dx = \int_0^1 e^{2(m-n)\pi ix}dx = \left[\frac{e^{2(m-n)\pi i x}}{2\pi i (m - n)}\right]_0^1 = 0,
		$$
		for $m \neq n$, they are orthogonal.
		\item For $n > 1$,
		\begin{align*}
			A_n
			&= \frac{2}{\pi}\int_0^\pi\cos x\sin nx dx = \frac{1}{\pi}\int_0^1[\sin (n+1)x + \sin(n-1)x]dx\\
			&= \left.\left(\frac{-\cos(n+1)x}{\pi(n + 1)} + \frac{-\cos (n-1)x}{\pi(n-1)}\right)\right|_0^\pi = \left(\frac{1 + (-1)^n}{\pi(n+1)} + \frac{1 + (-1)^n}{\pi(n - 1)}\right)\\
			&= \frac{(2n)(1 + (-1)^n)}{\pi(n^2 - 1)}.
		\end{align*}
		Since $A_1 = 0$, we have
		$$
		A_n =
		\begin{cases}
			0 & m-\text{odd},\\
			\frac{4n}{\pi(n^2 - 1)} & m - \text{even}.
		\end{cases}
		$$ 
		It converges to $\cos x$ for $0 < x < \pi$, to $-\cos x$ for $-\pi < x < 0$, and to zero for $x = -\pi,\ 0,\ \pi$.
		\item Separating variables we infer that there is a constant, denoted by $\lambda$ such that
		$$
		\frac{T_{tt}}{c^2T} = \frac{X_{xx}}{X} = -\lambda.
		$$
		Thus leads to the coupled ODE system
		\begin{align*}
			&\frac{d^2X}{dx^2} = -\lambda X,\quad 0 < x < L,\\
			&\frac{d^2T}{dx^2} = -\lambda c^2 T, \quad t > 0.
		\end{align*}
		Since $u$ is not the trivial solution, the boundary conditions imply $X(0) = X(L) = 0$. Thus, the function $X$ must satisfy the eigenvalue problem
		\begin{align*}
			\frac{d^2X}{dx^2} + \lambda X = 0\\
			X(0) = X(L) - 0.
		\end{align*}
		The solution to this eigenvalue problem is the infinite sequence
		$$
		X_n(x) = \sin\frac{n\pi x}{L},\quad \lambda_n = \left(\frac{n\pi}{L}\right)^2,\quad n\in \mathbb{N}.
		$$
		We proceed to the second equation for $T(t)$. Using the eigenvalues obtained above we find
		$$
		T_n(t) = \gamma_n \sin (\sqrt{\lambda_nc^2 t}) + \delta_n\cos(\sqrt{\lambda_nc^2t}),\quad n\mathbb{N}.
		$$
		We have thus derived the separated solutions
		$$
		u_n(x, t) = X_n(x)T_n(t) = \sin\frac{n\pi x}{L}\left(A_n\cos\frac{n\pi ct}{L} + B_n\sin\frac{n\pi ct}{L}\right),\quad n\in \mathbb{N}.
		$$
		Superposing these solutions we write
		$$
		u(x, t) = \sum_{n = 1}^\infty\left(A_n\cos\frac{n\pi ct}{L} + B_n\sin\frac{n\pi ct}{L}\right)\sin\frac{n\pi x}{L}.
		$$
		as the (generalized) solution to the problem of string vibrations with Dirichlet boundary conditions.\\
		It remains to find the coefficients $A_n$ and $B_n$. For this purpose we use the initial conditions
		$$
		A_n = \frac{2}{L}\int_0^Lf(x)\sin\frac{n\pi x}{L}dx,\quad B_n = \frac{2}{cn\pi}\int_0^Lg(x)\sin\frac{n\pi x}{L}dx,\quad n\geq 1.
		$$
		\item We substitute the initial conditions into the general solution
		$$
		u(x,t) = \sum_{n = 1}^\infty(A_n\cos nt + B_n\sin nt)\sin nx.
		$$
		We get
		$$
		u(x, 0) = \sum_{n = 1}^\infty A_n\sin nx = \sin^3x = -\frac{1}{4}\sin 3x + \frac{3}{4}\sin x,
		$$
		and
		$$
		\frac{\partial u(x, 0)}{\partial t} = \sum_{n = 1}^\infty nB_n\sin nx = \sin 2x.
		$$
		Hence,
		$$
		A_1 = -\frac{1}{4},\quad A_3 = \frac{3}{4},\quad B_2 = \frac{1}{2},
		$$
		and $A_n = 0$ if $n \neq 1,\ 3$; $B_n = 0$ if $n \neq 2$. We conclude that the formal solution is
		$$
		u(x, t) = -\frac{1}{4}\sin 3x\cos 3t + \frac{3}{4}\sin x\cos t + \frac{1}{2}\sin 2x\sin 2t.
		$$
		This is a finite sum of smooth functions and therefore is a classical solution.
		\item To obtain a homogeneous equation write $u = v + w$ where $w = w(t)$ satisfies
		$$
		w_t - kw_{xx} = A\cos \alpha t,\quad w(x, 0) \equiv 0.
		$$
		Therefore
		$$
		w(t) = \frac{A}{\alpha}\sin \alpha t.
		$$
		Note that $w$ satisfies also $w_x(0, t) = w_x(1, t) = 0$. Therefore, $v$ should solve
		\begin{align*}
			& v_t - kv_{xx} = 0,\quad 0 < x < 1,\ t > 0,\\
			& v_x(0, t) = v_x(1, t) = 0,\quad t\geq 0,\\
			& v(x, 0) = 1 + \cos^2\pi x,\quad 0 \leq x \leq 1.
		\end{align*}
		Thus,
		$$
		v(x, t) = \sum_{n = 0}^\infty B_ne^{-kn^2\pi^2t}\cos n\pi x = B_0\sum_{n = 0}^\infty B_ne^{-kn^2\pi^2t}\cos n\pi x.
		$$
		The coefficients $B_n$ are found to be
		$$
		B_0 = \int_0^1 [1 + \cos^2(\pi x)]dx = \frac{3}{2},\quad B_n = 2\int_0^1 [1 + \cos^2(\pi x)] \cos(n\pi x)dx,\quad n\geq 1.
		$$
		We obtain for $n \geq 1$:
		$$
		B_2 = \int_0^1\left(\frac{3}{2} + \cos(2\pi x)\right)\cos(2\pi x)dx = \frac{1}{2},\quad B_n = \int_0^1 \left(\frac{3}{2} + \cos(2\pi x)\right)\cos(n\pi x)dx = 0.
		$$
		Finally,
		$$
		u(x, t) = \frac{3}{2} + \frac{1}{2}\cos(2\pi x)e^{-4k\pi^2t} + \frac{A}{\alpha}\sin\alpha t.
		$$
		\item (A) Let $v(x, t) = u(x, t) - 1$. Then $v$ satisfies
		\begin{align*}
			& v_t = v_{xx},\\
			& v_x(0, t) = 0,\quad v(1, t) = 0,\\
			& v(x, 0) = x^2 - 1.
		\end{align*}
		The eigenvalues and eigenfunctions for the problem $-X^{\prime\prime} = \lambda X$ with IBC $X^\prime (0) = 0$ and $X(1) = 0$ are
		$$
		\lambda_n = \left(n + \frac{1}{2}\right)^2\pi^2,\quad x_n(x) = \cos\left[\left(n + \frac{1}{2}\right)\pi x\right],
		$$
		for $n \geq 0$. Thus
		$$
		u(x, t) = v(x, t) + 1 = 1 + \sum_{n = 0}^\infty A_n\cos\left[\left(n + \frac{1}{2}\right)\pi x\right]e^{-\left(n + \frac{1}{2}\right)^2\pi^2t},
		$$
		where
		$$
		A_n = 2\int_0^1 (x^2 - 1)\cos\left[\left(n + \frac{1}{2}\right)\pi x\right]dx = \frac{4(-1)^{n + 1}}{\left(n + \frac{1}{2}\right)^2\pi^3}.
		$$
		Therefore
		$$
		u(x, t) = 1 + \sum_{n = 0}^\infty \frac{4(-1)^{n + 1}}{\left(n + \frac{1}{2}\right)^3\pi^3}\cos\left[\left(n + \frac{1}{2}\right)\pi x\right]e^{-\left(n + \frac{1}{2}\right)^2\pi^2 t}.
		$$
		(B) As $t \to \infty$, every term in the series converges to zero, so the only term left is $1$.
		\item First note that the function
		$$
		v = -\frac{k}{2c}(x^2 - 2lx)
		$$
		satisfies the PDE and the boundary condition. Thus $w = u - v$ satisfies
		\begin{align*}
			& w_{tt} = c^2w_{xx},\\
			& w(0, t) = w_x(l, t) = 0,\\
			& w(x, 0) = \frac{k}{2c}(x^2 - 2lx),\quad w_t(x, 0) = V.
		\end{align*}
		The eigenfunctions of the homogeneous problem are $\sin(\beta nx)$, where $\beta_n = (n+1/2)\pi /l$ for $n \geq 0$. so we can write
		$$
		w(x, t) = \sum_{n = 0}^\infty \sin(\beta_n x)[A_n\cos(\beta_n ct) + B_n\sin (\beta_n ct)].
		$$
		The initial condition $w(x, 0) = \frac{c}{2k}(x^2 - 2lx)$ implies
		$$
		A_n = \frac{2}{l}\int_0^l\left(\frac{k}{2c}(x^2 - 2lx)\right)\sin(\beta_nx)dx = -\frac{2k}{lc^2\beta_n^3},
		$$
		and the condition $w_t(x, 0) = V$ implies
		$$
		B_n = \frac{2}{l}\int_0^l \frac{V}{c\beta_n}\sin(\beta_nx)dx = \frac{2V}{lc\beta_n^2}.
		$$
		Thus
		$$
		u(x, t) = -\frac{k}{2c}(x^2 - 2lx)\sum_{n = 0}^\infty\sin(\beta_nx)\left[-\frac{2k}{lc^2\beta_n^3}\cos(\beta_nct) + \frac{2V}{lc\beta_n^2}\sin(\beta_nct)\right].
		$$
		\item By d'Alember formula, we have
		\begin{align*}
			u(x, t)
			&= \frac{1}{2}[e^{x + ct} + e^{x - ct}] + \frac{1}{2c}\int_{x-ct}^{x + ct}\sin sds\\
			&= \frac{1}{2}e^x[e^{ct} + e^{-ct}] - \frac{1}{2c}[\cos(x + ct) - \cos(x-ct)]\\
			&= e^x\cosh(ct) + \frac{1}{c}\sin x\sin(ct).
		\end{align*}
		\item Factoring the operator yields
		$$
		(\partial_x + \partial_t)(\partial_x - 4\partial_t)u = 0.
		$$
		Set $v = u_x - 4u_t$. Then $v_x + v_t = 0$, so $v = h(x  t)$ and $u_x - 4u_t = h(x − t)$. One solution of the equation is $f(x − t)$, where $f^\prime(s) = h(s)/5$. The general solution of the homogeneous equation $u_x - 4u_t = 0$ is $g(4x + t)$, so
		$$
		u(x, t) = f(x − t) + g(4x + t).
		$$
		The initial conditions imply $f(x) + g(4x) = x^2$ and $-f^\prime (x) + g^\prime(4x) = e^x$. Differentiating the first equation leads to the system:
		\begin{align*}
			f^\prime (x) + 4g\prime (4x) &= 2x,\\
			-f^\prime (x) + g^\prime (4x) &= e^x.
		\end{align*}
		Solving this gives
		$$
		f^\prime(x) = \frac{2}{5}x - \frac{4}{5}e^x,\quad g^\prime(4x) = \frac{2}{5}x + \frac{1}{5}e^x.
		$$
		Thus
		$$
		f(x) = \frac{1}{5}x^2 - \frac{4}{5}e^x,\quad g(s) = \frac{1}{20}s^2 + \frac{1}{5}e^{s/4}.
		$$
		The solution is therefore
		\begin{align*}
			u(x, t)
			&= \frac{1}{5}(x - t)^2 - \frac{4}{5}e^{x-t} + \frac{1}{20}(4x + t)^2 + \frac{4}{5}e^{x + t/4}\\
			&= \frac{4}{5}[e^{x + t/4} - e^{x - t}] + x^2 + \frac{1}{4}t^2.
		\end{align*}
	\end{enumerate}
\end{document}