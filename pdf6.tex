\documentclass[11pt,a4paper,twoside]{article}
\usepackage[margin=1in, headheight=14pt]{geometry}
\usepackage{amsfonts,amsmath,amssymb,suetterl}
\usepackage{lmodern}
\usepackage[T1]{fontenc}
\usepackage{fancyhdr}
\usepackage{float}
\usepackage[utf8]{inputenc}
\usepackage{fontawesome}
\usepackage{enumerate}
\usepackage{xcolor}
\usepackage{hyperref}

\DeclareUnicodeCharacter{2212}{-}

\usepackage{mathrsfs}
\usepackage[nodisplayskipstretch]{setspace}

\setstretch{1.5}
\renewcommand{\footrulewidth}{0pt}

\parindent 0ex
\setlength{\parskip}{1em}
\pagestyle{empty}

\begin{document}
    %
	\begin{singlespace}
		\begin{center}
			\Huge Queen Mary\\
			\LARGE University of London
		\end{center}
		\Large \textbf{MTH5123} \hfill \Large \textbf{Differential Equations,} \hfill \Large \textbf{Autumn 2020}\\
		\large \textbf{Coursework 2 – Selected Solutions} \hfill \large \textbf{W. Huang}
		\rule{\textwidth}{0.4pt}
	\end{singlespace}
	\textbf{I. Practice Problems}\par
	\textbf{A.} Determine the general solutions of the following homogeneous linear differential equations. For each solution fix the constant of integration according to the given initial condition.
	%
	\begin{enumerate}[\bfseries 1)]
		\item $y^\prime = -xy,\ y(0) = -2:\ H(y) = \int \frac{dy}{y} = \ln |y|$, so $H^{-1}(u)=\pm e^u$ and on the right hand side $\int (-x)dx = -\frac{x^2}{2}+C$. The general solution is $y(x) = H^{-1}\left(-\frac{x^2}{2}+C\right) = \pm e^{-\frac{x^2}{2}+C} = De^{-\frac{x^2}{2}}$ with $D = \pm e^C$. The internal condition is $y(0) = D = -2$.
		\item $y^\prime = x\cos(x)y,\ y(0) = 1:\ H(y)=\int \frac{dy}{y} = \ln|y|$, so $H^{-1}(u) = \pm e^u$ and on the right-hand side $\int x\cos xdx = x\sin x + \cos x + C$ (integration by parts). The general solution is $y(x) = H^{-1}(x\sin x + \cos x + C) = \pm e^{x\sin x + \cos x + C} = De^{x\sin x + \cos x}$. The initial condition is $y(0) = De^1 = 1$, hence $D = e^{-1}$.
		\item $y^\prime = -y/(1+x),\ y(0) = -1:\ H(y) = \int \frac{dy}{y} = \ln|y|$, so $H^{-1}(u)=\pm e^u$ and on the right-hand side $-\int \frac{dx}{x+1} = -\ln|1+x|+C$. The general solution is $y(x) = H^{-1}(-\ln|x+1| + C) = \pm \frac{1}{|x+1|}e^C=\frac{D}{|x+1|}$. The initial condition is $y(0) = D/1 = -1$, hence $D = -1$.
		\item $y^\prime = y/(4-x^2),\ y(0) = 1:\ H(y) = \int \frac{dy}{y} = \ln|y|$, so $H^{-1}(u)=\pm e^u$ and on the right-hand side
		$$
		\int \frac{dx}{4-x^2}
		= \frac{1}{4}\int \frac{dx}{2-x}+\frac{1}{4}\int \frac{dx}{2+x}
		= -\frac{1}{4}\ln|2-x| + \frac{1}{4}\ln|2+x|+C
		= \frac{1}{4}\ln\frac{|2+x|}{|2-x|} + C.
		$$
		The general solution is
		$$
		y(x)
		= H^{-1}\left(\frac{1}{4}\ln \frac{|2+x|}{|2-x|}+C\right)
		= \pm e^{\frac{1}{4}\ln \frac{|2+x|}{|2-x|}+C}
		= D\left(\frac{|2+x|}{|2-x|}\right)^{1/4}
		$$
		with $D = \pm e^C$. The initial condition $y(0) = 1$ gives $D = 1$.
		\item $y^\prime = y/(x^2+2x+2),\ y(0) = 2:\ H(y)=\int \frac{dy}{y} = \ln |y|$, so $H^{-1}(u) = \pm e^u$ and the right-hand side $\int \frac{dx}{x^2+2x+2} = \int \frac{dx}{(x+1)^2+1}=\arctan (x+1) + C$. The general solution is $y(x) = H^{-1}(\arctan (x+1)+C) = De^{\arctan (x+1)}$. The initial condition is $y(0) = De^{\arctan 1} = De^{\pi/4} = 2$, hence $D= 2e^{-\pi/4}$.
	\end{enumerate}
	\textbf{Note:} All of the above ODEs are separable.\par
	%
	\textbf{B.} Solve the initial value problems associated with the following inhomogeneous linear differential equations.
	%
	\begin{enumerate}[\bfseries 1)]
		\item $y^\prime = y\frac{3x^2}{1+x^3} + x^2 + x^5,\ x>-1,\ y(0) = -1$: First we solve the homogeneous equation $y^\prime = y\frac{3x^2}{1+x^3}$ by separable variables. We have $H(y) = \int \frac{dy}{y} = \ln |y|,\ H^{-1}(u)=\pm e^u$ and on the right-hand side $\int \frac{3x^2}{1+x^3}dx = \int \frac{d(1+x^3)}{1+x^3}dx = \ln (x^3+1) + C$. The general solution is $y(x) = H^{-1}(\ln (x^3+1)+C) = \pm e^{\ln (x^3 + 1)+C} = D(x^3 + 1),\ x>-1$. We look for a solution to the inhomogeneous equation in the form of $y(x) = D(x)(x^3+1)$ so that differentiation gives $y^\prime = D^\prime(x^3+1)+3x^2D(x)= D^\prime (x^3+1) + y\frac{3x^2}{1+x^3}$. Comparing this with the right-hand side of the inhomogeneous equation leads to
		$$
		D^\prime(x^3+1) = x^2 + x^5,\ \Rightarrow \ D^\prime = \frac{x^2+x^5}{1+x^3} = x^2,\ D(x) = \frac{x^3}{3}+C
		$$
		so that the general solution to the inhomogeneous equation is given by $y(x) = \left(\frac{x^3}{3}+C\right)\left(x^3+1\right)$. Then $y(0) = C = -1$ and finally $y(x) = \left(\frac{x^3}{3}-1\right)\left(x^3+1\right)$.
		\item $y^\prime = -y\tan x + \cos x,\ -\pi/2 < x < \pi/2,\ y(0) = 2$: The general solution to the homogeneous equation $y^\prime = -y\tan x$ can be obtained by separation of variables. $H(y) = \ln |y|$, hence $H^{-1}(u) = \pm e^u$, and on the right-hand side $-\int \tan x dx = \ln |\cos x|+C$. This gives $y = \pm e^{\ln |\cos x|+C} = D\cos x,\ -\pi/2<x<\pi/2$, so we look for the solution to the inhomogeneous equation in the form $y(x) = D(x)\cos x$ which gives $y^\prime = D^\prime \cos x - D\sin x = D^\prime \cos x -D \tan x \cos x = D^\prime \cos x - y\tan x$. Comparing to the right-hand side of the inhomogeneous equation gives $D^\prime \cos x = \cos x$, then $D^\prime = 1$ and $D(x) = x+C$. Thus $y(x) = (x+C)\cos x$, which implies $y(0) = C = 2$.
	\end{enumerate}
	\textbf{Note:} These ODE’s are not separable, and one has to employ the variation of parameter method.\par
	%
	\textbf{C.} Determine the general solutions of the following inhomogeneous linear differential equations and solve the associated initial value problem.
	%
	\begin{enumerate}[\bfseries 1)]
		\item $y^\prime = 3y + 5,\ y(0) = -2$: Separating variables we have on the left-hand sind $H(y) = \int \frac{dy}{3y+5}=\frac{1}{3}\ln |3y+5|$ so that $H(y) = \frac{1}{3}\ln |3y+5| = u$ is solved as $y=\frac{1}{3}(\pm e^{3u}-5) = H^{-1}(u).$ On the right-hand side $\int dx = x+C$. The general solution is $y(x)=H^{-1}(x+C)=\frac{1}{3}(\pm e^{3(x+C)}-5) = \frac{1}{3}(De^{3x}-5)$ with $D=\pm e^{3C}$. The initial condition is $y(0) = \frac{D-5}{3} = -2$ which gives $D=-1$ and finally $y(x)=-\frac{1}{3}(e^{3x}+5)$.
		\item $y^\prime = -2xy+2x,\ y(0) = 0$: We rewrite the ODE as $y^\prime = 2x(y-1)$ and separating variables we have on the left-hand side $H(y)=\int \frac{dy}{y-1}=\ln |y-1|$ so that $H(y) = \ln |y-1| = u$ is solved as $y = (\pm e^u + 1) = H^{-1}(u)$. On the right-hand side $\int (-2x)dx + C = -x^2+C$. The general solution is $y(x)=H^{-1}(-x^2+C)=\left(\pm e^{-x^2 + C}+1\right) = De^{-x^2}+1.\ y(0) = 1+D=0$, hence $D=-1$ and the solution to the initial value problem is $y(x)=-e^{-x^2+1}$.
	\end{enumerate}
	%
	\textbf{Note:} Both ODEs above are separable, and it is easier to separate variables rather than to employ the variation of parameter method.\par

	\textbf{D}. Determine the general solutions to a linear inhomogeneous differential equation
	$$
	y^\prime = \frac{x}{1+x^2}y + \sqrt{\frac{1+x^2}{1-x^2}}
	$$
	\textbf{Solution:} We employ the variation of parameter method. First we solve the homogeneous equation $y^\prime = \frac{x}{1+x^2}y$ by separating variables. $H(y) = \ln |y|$, hence $H^{-1}(u) = \pm e^u,$ and on the right-hand side $\int \frac{x}{1+x^2}dx = \frac{1}{2}\ln (1+x^2)+C$. This gives $y = \pm e^{\frac{1}{2}\ln (1+x^2)+C} = D\sqrt{1+x^2}$. We look for a solution to the inhomogeneous equation in the form $y(x) = D(x)\sqrt{1+x^2}$ which gives
	$$
	y^\prime = D^\prime\sqrt{1+x^2}+y\frac{x}{1+x^2}
	$$
	Comparing to the right-hand side of the inhomogeneous equation in the form $y(x) = D(x)\sqrt{1+x^2}$ which gives
	$$
	D^\prime\sqrt{1+x^2} = \sqrt{\frac{1+x^2}{1-x^2}}
	$$
	so that $D^\prime = \frac{1}{\sqrt{1-x^2}}$ and
	$$
	D(x) = \int \frac{dx}{\sqrt{1-x^2}} = \arcsin x + C
	$$
	Thus finally $y(x) = (\arcsin x +C)\sqrt{1+x^2}$.\par
	\textbf{II.Homework Problems} \par
	Solutions will be discussed in Week 5, session 4 (see module session schedule in Qmplus as well).\par
	%
	\textbf{III. Method of Integrating Factors}\par
	\textbf{A. }A Walkthrough of the integrating factor method using the differential equation
	$$
	\frac{dy}{dx}+\frac{1}{2}y=\frac{1}{2}e^{x/3}.
	$$
	%
	\begin{enumerate}[\bfseries 1)]
		\item Use the variation of parameters method to solve the ODE with initial condition $y(0) = 1$. Be sure to identify this particular integral curve when you draw the family of solutions to the ODE in the $x \text{-} y$ plane.
		\item Multiplying the ODE by the function $\mu(x)$ gives
		$$
		\mu(x)\frac{dy}{dx}+\mu(x)\frac{1}{2}y=\mu(x)\frac{1}{2}e^{x/3}.
		$$
		Comparing the left hand side of this equations with the quantity
		$$
		\frac{d}{dx}[\mu(x)y] = \frac{d\mu}{dx}y+\mu\frac{dy}{dx},
		$$
		we see that the two expressions agree when $\frac{d\mu}{dx}=\mu(x)\frac{1}{2}$.
		\item Complete the following sentence: A function $f(x)$ whose derivative equals $\frac{1}{2}$ times the original function is given by $f(x)=e^{x/2}$. (Can you justify your answer without using separation of variables?)
		\item Plugging in $\mu(x) = Ce^{x/2}$ in the original ODE and simplifying gives the result
		$$
		\frac{d}{dx}[e^{x/2}y]=\frac{1}{2}e^{5x/6}.
		$$
		Integrate both sides of this equation to find the general solution
		$$
		y(x) = \frac{3}{5}e^{x/3} + Ce^{-x/2}.
		$$
		This solution should agree with the answer you found in the first question of this part, up to addition by some constant. (If your answers are not equal, you need to identify what the constant is!)
	\end{enumerate}
	%
	\textbf{B.} The Integrating Factor Method can be found in detail from the Reading List.
\end{document}