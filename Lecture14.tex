\documentclass[11pt,a4paper]{article}
\usepackage[margin=1in, headheight=14pt]{geometry}
\usepackage{amsfonts,amsmath,amssymb,suetterl}
\usepackage{lmodern}
\usepackage[T1]{fontenc}
\usepackage{fancyhdr}
\usepackage{float}
\usepackage[utf8]{inputenc}
\usepackage{fontawesome}
\usepackage{enumerate}
\usepackage{xcolor}
\usepackage{hyperref}
\usepackage{tikz}
\usepackage{nicefrac}
\usepackage{subcaption}
\usepackage{physics}

\DeclareUnicodeCharacter{2212}{-}

\usepackage{mathrsfs}
\usepackage[nodisplayskipstretch]{setspace}

\setstretch{1.5}
\renewcommand{\footrulewidth}{0pt}

\parindent 0ex
\setlength{\parskip}{1em}
\raggedbottom

\begin{document}
  %
	\begin{center}
		\vspace*{8cm}
		\Huge MA202 –Differential Equations\\
		\LARGE LECTURE 14
  \end{center}
  \newpage
  %
	\section*{Fundamental matrices}
	\begin{itemize}
		\item Suppose that $\vb{x}^{(1)}(t),\ \ldots,\ \vb{x}^{(n)}(t)$ form a fundamental set of solutions for $\vb{x}^\prime = P(t)\vb{x}$ on $\alpha < t <\beta$.
		\item The matrix
		$$
		\Psi (t)=
		\begin{pmatrix}
			x_1^{(1)}(t) & \ldots & x_1^{(n)}(t)\\
			\vdots & \ddots & \vdots\\
			x_n^{(1)}(t) & \ldots & x_n^{(n)}(t)
		\end{pmatrix},
		$$
		whose columns are $\vb{x}^{(1)}(t),\ \ldots,\ \vb{x}^{(n)}(t)$, is a fundamental matrix for the system $\vb{x}^\prime = P(t)$. This matrix is nonsingular since its columns are linearly independent, and hence $\det \Psi \neq 0$.
		\item Note also that since $\vb{x}^{(1)}(t),\ \ldots,\ \vb{x}^{(n)}(t)$ are solutions of $\vb{x}^\prime = p(t)\vb{x},\ \Psi$ satisfies the matrix differential equation $\Psi ^\prime = P(t)\Psi$.
	\end{itemize}
	%
	\section*{Example 1:}
	\begin{itemize}
		\item Consider the homogeneous equation $\vb{x}^\prime = A\vb{x}$ below
		$$
		\vb{x}^\prime =
		\begin{pmatrix}
			1 & 1\\
			4 & 1
		\end{pmatrix}
		$$
		\item The fundamental solutions for this system are:
		$$
		\vb{x}^{(1)}(t) =
		\begin{pmatrix}
			1\\
			2
		\end{pmatrix}e^{3t},\ 
		\vb{x}^{(2)}(t) =
		\begin{pmatrix}
			1\\
			-2
		\end{pmatrix}e^{-t}
		$$
		\item Thus a fundamental matrix for this system is
		$$
		\Psi(t) =
		\begin{pmatrix}
			e^{3t} & e^{-t}\\
			2e^{3t} & -2e^{-t}
		\end{pmatrix}
		$$
	\end{itemize}
	%
	\section*{Fundamental matrices and the general solution}
	\begin{itemize}
		\item The general solution of $\vb{x}^\prime = P(t)\vb{x}$
		$$
		\vb{x} = c_1\vb{x}^{(1)}(t) + \ldots + c_n\vb{x}^{(n)}
		$$
		can be expressed $\vb{x} = \Psi (t)\vb{c}$, where $\vb{c}$ is a constant vector with components $c_1,\ \ldots,\ c_n$:
		$$
		\vb{x} = \Psi(t)\vb{c}=
		\begin{pmatrix}
			x_1^{(1)}(t) & \ldots & x_1^{(n)}(t)\\
			\vdots & \ddots & \vdots\\
			x_n^{(1)}(t) & \ldots & x_n^{(n)}(t)
		\end{pmatrix}
		\begin{pmatrix}
			c_1\\
			\vdots\\
			c_n
		\end{pmatrix}
		$$
	\end{itemize}
	%
	\section*{Fundamental matrix and the initial value problem}
	\begin{itemize}
		\item Consider an initial value problem
		$$
		\vb{x}^\prime = P(t)\vb{x},\ \vb{x}(t_0) = \vb{x}^0
		$$
		where $\alpha < t_0 < \beta$ and $\vb{x}^0$ is a given initial vector.
		\item Now the solution has the form $\vb{x} = \Psi (t)\vb{c}$, hence we choose $\vb{c}$ so as to satisfy $\vb{x}(t_0) = \vb{x}^0$. 
		\item Recalling $\Psi (t_0)$ is nonsingular, it follows that
		$$
		\Psi(t_0)\vb{c} = \vb{x^0} \Rightarrow \vb{c} = \Psi^{-1}(t_0)\vb{x}^0
		$$
		\item Thus our solution $\vb{x} = \Psi(t)\vb{c}$ can be expressed as
		$$
		\vb{x} = \Psi(t)\Psi^{-1}(t_0)\vb{x^0}
		$$
	\end{itemize}
	%
	\section*{Recall: Theorem 7.4.4}
	\begin{itemize}
		\item Let
		$$
		\vb{e}^{1} =
		\begin{pmatrix}
			1\\
			0\\
			0\\
			\vdots \\
			0
		\end{pmatrix},\ 
		\vb{e}^{2} =
		\begin{pmatrix}
			0\\
			1\\
			0\\
			\vdots \\
			0
		\end{pmatrix},\ \ldots,\ 
		\vb{e}^{n} =
		\begin{pmatrix}
			0\\
			0\\
			0\\
			\vdots \\
			1
		\end{pmatrix}
		$$
		\item Let $\vb{x}^{(1)},\ \ldots,\  \vb{x}^{(n)}$ be solutions of $\vb{x}^\prime = P(t)\vb{x}$ on $I: \alpha < t < \beta$ that satisfy the initial conditions
		$$
		\vb{x}^{(1)}(t_0) = \vb{e}^{(1)},\ \ldots,\ \vb{x}^{(n)}(t_0) = \vb{e}^{(n)},\ \alpha<t_0<\beta
		$$
		Then $\vb{x}^{(1)},\ \ldots,\ \vb{x}^{(n)}$ are fundamental solutions of $\vb{x}^\prime = P(t)\vb{x}$.
	\end{itemize}
	%
	\section*{Fundamental matrix and theorem 7.4.4}
	\begin{itemize}
		\item Suppose $\vb{x}^{(1)}(t),\ \ldots,\ \vb{x}^{(n)}(t)$ form the fundamental solutions given by Thm 7.4.4. Denote the corresponding fundamental matrix by $\Phi(t)$. Then columns of $\Phi(t)$ are $\vb{x}^{(1)}(t),\ \ldots,\ \vb{x}^{(n)}(t)$, and hence
		$$
		\Phi(t_0) = 
		\begin{pmatrix}
			1 & 0 & \ldots & 0\\
			0 & 1 & \ldots & 0\\
			\vdots & \vdots & \ddots & \ldots\\
			0 & 0 & \ldots & 1
		\end{pmatrix} = \vb{I}
		$$
		\item Thus $\Phi^{-1}(t_0) = I$, and the hence the general solution to the corresponding initial value problem is
		$$
		\vb{x} = \Phi(t)\Phi^{-1}(t_0)\vb{x}^0 = \Phi(t)\vb{x}^0
		$$
		\item It follows that for any fundamental matrix $\Phi(t)$,
		$$
		\vb{x} = \Psi(t)\Psi^{-1}(t_0)\vb{x}^0 = \Phi(t)\vb{x}^0\Rightarrow \Phi(t) = \Psi(t)\Psi^{-1}(t_0)
		$$
	\end{itemize}
	%
	\section*{The fundamental matrix $\Phi$ and varying initial conditions}
	\begin{itemize}
		\item Thus when using the fundamental matrix $\Phi(t)$, the general solution to an IVP is (noting that $\Phi^{-1}(t_0)$ is the unit matrix)
		$$
		\vb{x} = \Phi(t)\Phi^{-1}(t_0)\vb{x}^0 = \Phi(t)\vb{x}^0
		$$
		\item This representation is useful if same system is to be solved for many different initial conditions, such as a physical system that can be started from many different initial states.
		\item Also, once $\Phi(t)$ has been determined, the solution to each set of initial conditions can be found by matrix multiplication, as indicated by the equation above.
		\item Thus $\Phi(t)$ represents a linear transformation of the initial conditions $\vb{x}^0$ into the solution $\vb{x}(t)$ at time $t$.
	\end{itemize}
	%
	\section*{Example 2: Find $\Phi(t)$ for $2 \times 2$ system (1 of 5)}
	\begin{itemize}
		\item Find $\Phi(t)$ such that $\Phi(0) = I$ for the system below.
		$$
		\vb{x}^\prime = 
		\begin{pmatrix}
			1 & 1\\
			4 & 1
		\end{pmatrix}\vb{x}
		$$
		\item Solution: First, we must obtain $\vb{x}^{(1)}(t)$ and $\vb{x}^{(2)}(t)$ such that
		$$
		\vb{x}^{(1)}(0) = 
		\begin{pmatrix}
			1\\
			0
		\end{pmatrix},\ 
		\vb{x}^{(2)}(0) =
		\begin{pmatrix}
			0\\
			1
		\end{pmatrix}
		$$
		\item We know from previous results that the general solution is
		$$
		\vb{x} = c_1
		\begin{pmatrix}
			1\\
			2
		\end{pmatrix} + c_2
		\begin{pmatrix}
			1\\
			-2
		\end{pmatrix}e^{-t}
		$$
		\item Every solution can be expressed in terms of the general solution, and we use this fact to find $\vb{x}^{(1)}(t)$ and $\vb{x}^{(2)}(t)$.
	\end{itemize}
	%
	\section*{Example 2: Use general solution (2 of 5)}
	\begin{itemize}
		\item Thus, to find $\vb{x}^{(1)}(t)$, express it in terms of the general solution
		$$
		\vb{x}^{(1)}(t) = c_1
		\begin{pmatrix}
			1\\
			2
		\end{pmatrix}e^{3t} + c_2
		\begin{pmatrix}
			1\\
			-2
		\end{pmatrix}e^{-t}
		$$
		and then find the coefficients $c_1$ and $c_2$. 
		\item To do so, use the initial conditions to obtain
		$$
		\vb{x}^{(1)}(0) = c_1
		\begin{pmatrix}
			1\\
			2
		\end{pmatrix} + c_2
		\begin{pmatrix}
			1\\
			-2
		\end{pmatrix}=
		\begin{pmatrix}
			1\\
			0
		\end{pmatrix}
		$$
		or equivalently,
		$$
		\begin{pmatrix}
			1 & 1\\
			2 & -2
		\end{pmatrix}
		\begin{pmatrix}
			c_1\\
			c_2
		\end{pmatrix}=
		\begin{pmatrix}
			1\\
			0
		\end{pmatrix}
		$$
	\end{itemize}
	%
	\section*{Example 2: Solve for $\vb{x}^{(1)}(t)$ (3 of 5)}
	\begin{itemize}
		\item To find $\vb{x}^{(1)}(t)$, we therefore solve
		$$
		\begin{pmatrix}
			1 & 1\\
			2 & -2
		\end{pmatrix}
		\begin{pmatrix}
			c_1\\
			c_2
		\end{pmatrix}=
		\begin{pmatrix}
			1\\
			0
		\end{pmatrix}
		$$
		by row reducing the augmented matrix:\\
		$
		\begin{pmatrix}
			1 & 1 & 1\\
			2 & -2 & 0
		\end{pmatrix}\to
		\begin{pmatrix}
			1 & 1 & 1\\
			0 & -4 & 0
		\end{pmatrix}\to
		\begin{pmatrix}
			1 & 1 & 1\\
			0 & 1 & 1/2
		\end{pmatrix}\to
		\begin{pmatrix}
			1 & 0 & 1/2\\
			0 & 1 & 1/2
		\end{pmatrix}\to
		\begin{matrix}
			c_1 & & =1/2\\
			& c_2 & =1/2
		\end{matrix}
		$
		\item Thus
		$$
		\vb{x}^{(1)}(t) = \frac{1}{2}
		\begin{pmatrix}
			1\\
			2
		\end{pmatrix}e^{3t} + \frac{1}{2}
		\begin{pmatrix}
			1\\
			-2
		\end{pmatrix}e^{-t}=
		\begin{pmatrix}
			\frac{1}{2}e^{3t} + \frac{1}{2}e^{-t}\\
			e^{3t} - e^{-t}
		\end{pmatrix}
		$$
	\end{itemize}
	\section*{Example 2: Solve for $\vb{x}^{(2)}(t)$ (4 of 5)}
	\begin{itemize}
		\item To find $\vb{x}^{(2)}(t)$, we similarly solve
		$$
		\begin{pmatrix}
			1 & 1\\
			2 & -2
		\end{pmatrix}
		\begin{pmatrix}
			c_1\\
			c_2
		\end{pmatrix}=
		\begin{pmatrix}
			1\\
			0
		\end{pmatrix}
		$$
		by row reducing the augmented matrix\\
		$
		\begin{pmatrix}
			1 & 1 & 0\\
			2 & -2 & 1
		\end{pmatrix}\to
		\begin{pmatrix}
			1 & 1 & 0\\
			0 & -4 & 1
		\end{pmatrix}\to
		\begin{pmatrix}
			1 & 1 & 0\\
			0 & 1 & -1/4
		\end{pmatrix}\to
		\begin{pmatrix}
			1 & 0 & 1/4\\
			0 & 1 & -1/4
		\end{pmatrix}\to
		\begin{matrix}
			c_1 & & =1/4\\
			& c_2 & =-1/4
		\end{matrix}
		$
		\item Thus
		$$
		\vb{x}^{(2)}(t) = \frac{1}{4}
		\begin{pmatrix}
			1\\
			2
		\end{pmatrix}e^{3t} - \frac{1}{4}
		\begin{pmatrix}
			1\\
			-2
		\end{pmatrix}e^{-t}=
		\begin{pmatrix}
			\frac{1}{4}e^{3t} - \frac{1}{4}e^{-t}\\
			\frac{1}{2}e^{3t} + \frac{1}{2}e^{-t}
		\end{pmatrix}
		$$
	\end{itemize}
	\section*{Example 2: Obtain $\Phi(t)$ (5 of 5)}
	\begin{itemize}
		\item The columns of $\Phi(t)$ are given by $\vb{x}^{(1)}(t)$ and $\vb{x}^{(2)}(t)$, and thus from the previous slide we have
		$$
		\Phi(t) =
		\begin{pmatrix}
			\frac{1}{2}e^{3t} + \frac{1}{2}e^{-t} & \frac{1}{4}e^{3t} - \frac{1}{4}e^{-t}\\
			e^{3t} - e^{-t} & \frac{1}{2}e^{3t} + \frac{1}{2}e^{-t}
		\end{pmatrix}
		$$
		\item Note $\Phi(t)$ is more complicated than $Psi(t)$ found in Ex 1. However, now that we have $\Phi(t)$, it is much easier to determine the solution to any set of initial conditions.
		$$
		\Psi(t) =
		\begin{pmatrix}
			e^{3t} & e^{-t}\\
			2e^{3t} & -2e^{-t}
		\end{pmatrix}
		$$
	\end{itemize}
	%
	\section*{Matrix exponential functions}
	\begin{itemize}
		\item Consider the following two cases:
		\begin{itemize}
			\item[\labelitemi] The solution to $x^\prime = ax,\ x(0) = x_0$, is $x = x_0e^{at}$, where $e^0 = 1$.
			\item[\labelitemi] The solution to $\vb{x}^\prime = A\vb{x},\ \vb{x}(0) = \vb{x}^0$, is $\vb{x} = \Phi(t)\vb{x}^0$, where $\Phi(0) = I$. 
		\end{itemize}
		\item Comparing the form and solution for both of these cases, we might expect $\Phi(t)$ to have an exponential character.
		\item Indeed, it can be shown that $\Phi(t) = e^{At}$, where
		$$
		e^{At} = \sum_{n=0}^{\infty}\frac{A^nt^n}{n!} = I+\sum_{n=1}^{\infty}\frac{A^nt^n}{n!}
		$$
		is a well defined matrix function that has all the usual properties of an exponential function. See text for details.
		\item Thus the solution to $\vb{x}^\prime = A\vb{x},\ \vb{x}(0) = \vb{x}^0$, is $\vb{x} = e^{At}\vb{x}^0$. 
	\end{itemize}
	%
	\section*{Example 3: Matrix exponential function}
	\begin{itemize}
		\item Consider the diagonal matrix $A$ below
		$$
		A =
		\begin{pmatrix}
			1 & 0\\
			0 & 2
		\end{pmatrix}
		$$
		\item Then
		$$
		A^2 = 
		\begin{pmatrix}
			1 & 0\\
			0 & 2
		\end{pmatrix}
		\begin{pmatrix}
			1 & 0\\
			0 & 2
		\end{pmatrix} = 
		\begin{pmatrix}
			1 & 0\\
			0 & 2^2
		\end{pmatrix},\ A^3=
		\begin{pmatrix}
			1 & 0\\
			0 & 2^2
		\end{pmatrix}
		\begin{pmatrix}
			1 & 0\\
			0 & 2
		\end{pmatrix} =
		\begin{pmatrix}
			1 & 0\\
			0 & 2^3
		\end{pmatrix},\ldots
		$$
		\item In general,
		$$
		A^n =
		\begin{pmatrix}
			1 & 0\\
			0 & 2^n
		\end{pmatrix}
		$$
		\item Thus
		$$
		e^{At} = \sum_{n=0}^{\infty}\frac{A^nt^n}{n!} = \sum_{n=0}^\infty
		\begin{pmatrix}
			1/n! & 0\\
			0 & 2^n/n!
		\end{pmatrix}t^n =
		\begin{pmatrix}
			e^t & 0\\
			0 & e^{2t}
		\end{pmatrix}
		$$
	\end{itemize}
	%
	\section*{Coupled systems of equations}
	\begin{itemize}
		\item Recall that our constant coefficient homogeneous system
		$$
		\begin{matrix}
			x_1^\prime = a_{11}x_1 + a_{12}x_2 + \ldots + a_{1n}x_n\\
			\vdots\\
			x_n^\prime = a_{n1}x_1 + a_{n2}x_2 + \ldots + a_{nn}x_n,
		\end{matrix}
		$$
		written as $\vb{x}^\prime = A\vb{x}$ with
		$$
		\vb{x}(t) = 
		\begin{pmatrix}
			x_1(t)\\
			\vdots\\
			x_n(t)
		\end{pmatrix},\ A =
		\begin{pmatrix}
			a_{11} & \ldots & a_{1n}\\
			\vdots & \ddots & \vdots\\
			a_{n1} & \ldots & a_{nn}
		\end{pmatrix},
		$$
		is a system of coupled equations that must be solved simultaneously to find all the unknown variables. 
	\end{itemize}
	%
	\section*{Uncoupled systems and diagonal matrices}
	\begin{itemize}
		\item In contrast, if each equation had only one variable, solved for independently of other equations, then task would be easier.
		\item In this case our system would have the form
		$$
		\begin{matrix}
			x_1^\prime = d_{11} x_1 + 0x_2 +\ldots + 0x_n\\
			x_2^\prime = 0 x_1 + d_{22}x_2 +\ldots + 0x_n\\
			\vdots\\
			x_n^\prime = 0 x_1 + 0x_2 +\ldots + d_{nn}x_n,
		\end{matrix}
		$$
		or $\vb{x}^\prime = \vb{D}\vb{x}$, where $\vb{D}$ is a diagonal matrix:
		$$
		\vb{x}(t) =
		\begin{pmatrix}
			x_1(t)\\
			\vdots\\
			x_n(t)
		\end{pmatrix},\ \vb{D} =
		\begin{pmatrix}
			d_{11} & 0 & \ldots & 0\\
			0 & d_{22} & \ldots & 0\\
			\vdots & \vdots & \ddots & \vdots\\
			0 & 0 & \vdots & d_{nn}
		\end{pmatrix}
		$$ 
	\end{itemize}
	%
	\section*{Uncoupling: Transform matrix $\vb{T}$}
	\begin{itemize}
		\item In order to explore transforming our given system $\vb{x}^\prime = A\vb{x}$ of coupled equations into an uncoupled system $\vb{x}^\prime = \vb{D}\vb{x}$, where $\vb{D}$ is a diagonal matrix, we will use the eigenvectors of $A$.
		\item Suppose $A$ is $n \times n$ with $n$ linearly independent eigenvectors $\xi^{(1)},\ \ldots,\ \xi^{(n)}$, and corresponding eigenvalues $\lambda_1,\ \ldots,\ \lambda_2$.
		\item Define $n \times n$ matrices $\vb{T}$ and $\vb{D}$ using the eigenvalues \& eigenvectors of $A$:
		$$
		\vb{T}=
		\begin{pmatrix}
			\xi_1^{(1)} & \ldots & \xi_1^{(n)}\\
			\vdots & \ddots & \vdots\\
			\xi_n^{(1)} & \ldots & \xi_n^{(n)}
		\end{pmatrix},\ \vb{D} =
		\begin{pmatrix}
			\lambda_1 & 0 & \ldots & 0\\
			0 & \lambda_2 & \ldots & 0\\
			\vdots & \vdots & \ddots & \vdots\\
			0 & 0 & \ldots & \lambda_n
		\end{pmatrix}
		$$
		\item Note that $\vb{T}$ is nonsingular, and hence $\vb{T}^{-1}$ exists.
	\end{itemize}
	%
	\section*{Uncoupling: $\vb{T}^{-1}A\vb{T} = \vb{D}$}
	\begin{itemize}
		\item Recall here the definitions of $A, \vb{T}$ and $\vb{D}$:
		$$
		A =
		\begin{pmatrix}
			a_{11} & \ldots & a_{1n}\\
			\vdots & \ddots & \vdots\\
			a_{n1} & \ldots & a_{nn}
		\end{pmatrix},\ 
		\vb{T}=
		\begin{pmatrix}
			\xi_1^{(1)} & \ldots & \xi_1^{(n)}\\
			\vdots & \ddots & \vdots\\
			\xi_n^{(1)} & \ldots & \xi_n^{(n)}
		\end{pmatrix},\ 
		\vb{D} =
		\begin{pmatrix}
			\lambda_1 & 0 & \ldots & 0\\
			0 & \lambda_2 & \ldots & 0\\
			\vdots & \vdots & \ddots & \vdots\\
			0 & 0 & \ldots & \lambda_n
		\end{pmatrix}
		$$
		\item Then the columns of $A\vb{T}$ are $A\xi^{(1)},\ \ldots,\ A\xi^{(n)}$, and hence
		$$
		A\vb{T} =
		\begin{pmatrix}
			\lambda_1\xi_1^{(1)} & \ldots & \lambda_n\xi_1^{(n)}\\
			\vdots & \ddots & \vdots\\
			\lambda_1\xi_n^{(1)} & \ldots & \lambda_n\xi_n^{(n)}
		\end{pmatrix} = \vb{T}\vb{D}
		$$
		\item It follows that $\vb{T}^{-1}A\vb{T} = \vb{D}$.
	\end{itemize}
	%
	\section*{Similarity transformations}
	\begin{itemize}
		\item Thus, if the eigenvalues and eigenvectors of $A$ are known, then $A$ can be transformed into a diagonal matrix $\vb{D}$, with
		$$
		\vb{T}^{-1}A\vb{T} = \vb{D}
		$$
		\item This process is known as a \textbf{similarity transformation}, and $A$ is said to be \textbf{similar} to $\vb{D}$. Alternatively, we could say that $\vb{A}$ is \textbf{diagonalizable}.
		$$
		A =
		\begin{pmatrix}
			a_{11} & \ldots & a_{1n}\\
			\vdots & \ddots & \vdots\\
			a_{n1} & \ldots & a_{nn}
		\end{pmatrix},\ 
		\vb{T}=
		\begin{pmatrix}
			\xi_1^{(1)} & \ldots & \xi_1^{(n)}\\
			\vdots & \ddots & \vdots\\
			\xi_n^{(1)} & \ldots & \xi_n^{(n)}
		\end{pmatrix},\ 
		\vb{D} =
		\begin{pmatrix}
			\lambda_1 & 0 & \ldots & 0\\
			0 & \lambda_2 & \ldots & 0\\
			\vdots & \vdots & \ddots & \vdots\\
			0 & 0 & \ldots & \lambda_n
		\end{pmatrix}
		$$
	\end{itemize}
	%
	\section*{Example 4: Find the transformation matrix $\vb{T}$ (1 of 2)}
	\begin{itemize}
		\item For the matrix $A$ below, find the similarity transformation matrix $\vb{T}$ and show that $A$ can be diagonalized.
		$$
		A = 
		\begin{pmatrix}
			1 & 1\\
			4 & 1
		\end{pmatrix}
		$$
		\item We already know that the eigenvalues are $\lambda_1 = 3,\ \lambda_2 = -1$ with corresponding eigenvectors
		$$
		\xi^{(1)}(t) =
		\begin{pmatrix}
			1\\
			2
		\end{pmatrix},\ \xi^{(2)}(t) =
		\begin{pmatrix}
			1\\
			-2
		\end{pmatrix}
		$$
		\item Thus
		$$
		\vb{T} =
		\begin{pmatrix}
			1 & 1\\
			2 & -2
		\end{pmatrix},\ \vb{D} =
		\begin{pmatrix}
			3 & 0\\
			0 & -1
		\end{pmatrix}
		$$
	\end{itemize}
	%
	\section*{Example 4: Similarity transformation (2 of 2)}
	\begin{itemize}
		\item To find $\vb{T}^{-1}$, augment the identity to $\vb{T}$ and row reduce:\\
		$
		\begin{pmatrix}
			1 & 1 & 1 & 0\\
			2 & -2 & 0 & 1
		\end{pmatrix}\to
		\begin{pmatrix}
			1 & 1 & 1 & 0\\
			0 & -4 & -2 & 1
		\end{pmatrix}\to
		\begin{pmatrix}
			1 & 1 & 1 & 0\\
			0 & 1 & 1/2 & -1/4
		\end{pmatrix}\to
		\begin{pmatrix}
			1 & 0 & 1/2 & 1/4\\
			0 & 1 & 1/2 & -1/4
		\end{pmatrix}\to \vb{T}^{-1}
		\begin{pmatrix}
			1/2 & 1/4\\
			1/2 & -1/4
		\end{pmatrix}
		$
		\item Then
		\begin{align*}
			\vb{T}^{-1}A\vb{T} &=
			\begin{pmatrix}
				1/2 & 1/4\\
				1/2 & -1/4
			\end{pmatrix}
			\left[
				\begin{pmatrix}
					1 & 1\\
					4 & 1
				\end{pmatrix}
				\begin{pmatrix}
					1 & 1\\
					2 & -2
				\end{pmatrix}
			\right]\\
			&=
			\begin{pmatrix}
				1/2 & 1/4\\
				1/2 & -1/4
			\end{pmatrix}
			\begin{pmatrix}
				3 & -1\\
				6 & 2
			\end{pmatrix} =
			\begin{pmatrix}
				3 & 0\\
				0 & -1
			\end{pmatrix} = \vb{D}
		\end{align*}
		\item Thus $A$ is similar to $\vb{D}$, and hence $A$ is diagonalizable.
	\end{itemize}
\end{document}